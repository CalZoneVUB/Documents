\section{Vergadering \MeetingDate}
\subsection{Praktische informatie}
\subsubsection{Tijd \& Plaats}
Uur: 15:00 - 17:00
\\
Lokaal: E0.04
\subsubsection{Aanwezigheidslijst}
\begin{table}[htbp]
	\centering
	\begin{tabular}{c|c}
		\multirow{2}{*}{\textbf{Aanwezig}} & Tim Witters, Youri Coppens, Pieter Meiresone, \\
		& Nicolas Carraggi,  Christophe Gaethofs, Fernando Suarez\\
		\hline
		\textbf{Afwezig} & Sam van den Vonder \\
	\end{tabular}
\end{table}

\subsubsection{Verantwoordelijkheden}
\begin{itemize}
	\item Voorzitter: Pieter Meiresone
	\item Secretaris: Tim Witters
\end{itemize}

\subsubsection{Kritiekpunten op verslag vorige vergadering}
Geen kritiekpunten.

\subsection{Vooruitgang sinds vorige vergadering} \label{sec:vooruitgangSindsVorigeVergadering}
	\begin{itemize}
		\item Fernando: In SRS functionele requirements over specifieke mobiele functionaliteit uitgeschreven. Feeback van eerste iteratie moet nog steeds aangepast worden.
		\item Sam: test document en extra test voorbeeld gemaakt. Dit zal later nog worden uitgebreid voor verwarring en goed verloop van het testen te genereren. Verder zijn de SDD aanpassingen gebeurd en zal opgevolgd worden als hier veranderingen aan gebeuren.
		\item 
		{
			Nicolas: Analyse van  de datadump is ongeveer gedaan. De voornaamste inhoud van de datadump:
			\begin{itemize}
				\item Inhoud van de verschillende studieprogramma's.
				\item Data over de verschillende professoren.
			\end{itemize}
		}
		\item Pieter: SPMP aanpassingen: Extra activiteiten zijn toegevoegd voor Gantt chart en maken van Gantt chart voor komende iteraties. 
	\end{itemize}
	
\subsection{Vergaderingspunten}
Tijdens deze vergadering dienen volgende vergaderingspunten besproken te worden.
\begin{table} [H]
	\centering
	\begin{tabular} {l|c|c|c}
		Onderwerp & Tijdsduur & Prioriteit & Beschrijving \\ % Prioriteit: Hoog, Middelmatig of Laag
		\hline
		Bespreking documenten 		& 30min & Middelmatig &  \\
		Coding Implementatie & 30min & Hoog & \\
	\end{tabular}
\end{table}

\subsubsection{Bespreking documenten}
Zie sectie \ref{sec:vooruitgangSindsVorigeVergadering}.

\subsubsection{Bespreking van implementatie}
Volgende zaken zijn besproken:
\begin{itemize}
	\item Vakken ophalen van de datadump.
	\item Lokalen zullen worden opgehaald via een scraper.
	\item Activation keys mogen maar 1 bezitten per email.
	\item Professors moet de mogelijkheid krijgen om bij het vak een assistent toe te voegen. Als dit over een niet assistent gaat, dan krijgt de programmabeheerder deze informatie op een speciale pagina voorgeschoteld. Hierbij kan hij dit automatisch promoveren van de student. Dit moet nog toegevoegd worden aan SRS.
	\item De Admin moet een nieuwe admin kunnen toevoegen.
\end{itemize}

Codering die gebeurd moet zijn tegen volgende vergadering
\begin{enumerate}
	\item Lokalen
	\item Vakken HOC / WPO
	\item Entry (Vak op bepaald uur in bepaald lokaal)
	\item Entry tonen op webpagina
	\item Profielpagina webpagina
	\item Beveiliging (/public and /private /private/edit)
\end{enumerate}

\subsection{Volgende vergadering}
De volgende vergadering vindt plaats op Dinsdag 25/02/2014 12u - 14u, verzamelen aan IG.
Coding sessie zal plaats vinden op Maandag 24/02/2014 - 14u
\subsubsection{Activiteiten tegen volgende vergadering} \label{sec:TODOActiviteiten}
\begin{table} [H]
	\centering
	\begin{tabular} {l|l|l}
		\textbf{Activiteit} & \textbf{Team} & \textbf{Opmerkingen} \\
		\hline
		SRS volledig afwerken & Fernando & geen \\
		STD afwerken en voorbeeld testen schrijven & Sam & geen \\
		Lokalen,vakken,entry & Pieter & geen \\
		Beveiliging en user paginas & Tim & geen \\
		Profielpagina en kalender & Christophe & geen \\
		Scraper lokaal scraper & Youri & geen \\
		User keys herzenden + deleten oude activation key & Nikolas & geen \\
	\end{tabular}
\end{table}

Verder zijn er reeds volgende activiteiten voor tijdens de coding sessie:
\begin{itemize}
	\item Nikolas: Alle data opslaan in database
	\item Nikolas: Datadump parsen
\end{itemize}

\subsubsection{Agenda Punten volgende vergadering}
Nog geen vergaderingspunten. Dit zal bepaald worden na de coding sessie.