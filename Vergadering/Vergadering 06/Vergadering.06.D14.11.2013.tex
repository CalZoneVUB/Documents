\section{Vergadering 14/11/2013}
\subsection{Aanwezigheidslijst}
\begin{table}[htbp]
	\centering
	\begin{tabular}{c|c}
		\multirow{2}{*}{\textbf{Aanwezig}} & Tim Witters, Youri Coppens, Pieter Meiresone, \\
		& Nicolas Carraggi,  Christophe Gaethofs, Sam van den Vonder \\
		\hline
		\textbf{Afwezig} & Fernando Suarez Groen\\
	\end{tabular}
\end{table}

\subsection{Verantwoordelijkheden}
\begin{itemize}
	\item Voorzitter: Pieter Meiresone
	\item Secretaris: Youri Coppens 
\end{itemize}

\subsection{Vooruitgang sinds vorige vergadering}
\subsection{Vergaderingspunten}
Tijdens deze vergadering dienen volgende vergaderingspunten besproken te worden.
\begin{table} [H]
	\centering
	\begin{tabular}{l|c|c|c}
		Onderwerp & Tijdsduur & Prioriteit & Beschrijving \\
		\hline
		Time Tracking & 2min & Hoog & \\
		Leden Pagina & 2min & Laag & \\
		Documenten Template & 2min & Middelmatig & Logo ontbreekt nog op de documenten\\
		SRS Progressie & 15min & Hoog & \\
		SPMP Progressie & 15min & Hoog & \\
		SCMP Briefing & 15min & Hoog & \\
		Werkplanning opstellen & 15min & Middelmatig & \\
	\end{tabular}
\end{table}

\begin{itemize}
\item SRS progress;
Tim heeft onderverdeling herzien: user requirements en systeemvereisten
binnen systeemvereisten zijn is er een sectie over user management

Pieter: 
! Tim en Youri moeten files goedzetten, requirements duidelijker groeperen en verder aanvullen daar waar nodig 
Youri beschrijft hoe hij het stappenplan aanneemt: gericht op de view van de gebruikers.
Pieter vindt dit goed zodat er screenshots bijkunnen:
! Fernando maakt screenshots in PHOTOSHOP ter illustratie bij de stappenplannen

\item SPMP progress:
Pieter heeft een paar puntjes van de feedback aan het behandelen.
Eerst metrieken, want dit was vrij leeg.
Gebruikt een Eclipse plugin voor metrieken: methoden, klassen etc. 
Risico management uitgebreid. 
Er wordt gebrainstormt over extra risico's, want tot nu toe zijn er weinig. Na enkele minuten conclusie dat we beter 's avonds doen.
! Er wordt gebrainstormt over risico's
! Opmerkingen van de feedback verwerken

\item SCMP briefing:
Christophe legt uit wat hij voor de SCMP al heeft gedaan. 
Korte uitleg over wie het plan beheert 
Activiteiten: hoe werken op git en website. Dient aanpassing te verkrijgen omdat ShareLaTeX verworpen is door de groep.
Issues en milestones: vereist nog een bespreking.
! 
Groei en planning: aanpassingen van het document enzo
Resources: waarom github en andere software(eclipse, java...)
Workflow op github. 
! Procedures moeten in de SCMP komen (how to git enzo)
! Iedereen dient het SCMP te lezen als richtlijn voor de werking

! Overlopen SQAP

Einde vergadering


\end{itemize}
\subsection{Action items tegen volgende vergadering}
\subsection{Agenda Punten volgende vergadering}

