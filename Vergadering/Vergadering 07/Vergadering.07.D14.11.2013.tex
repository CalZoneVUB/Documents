\section{Vergadering 14/11/2013}
\subsection{Aanwezigheidslijst}
\begin{table}[htbp]
	\centering
	\begin{tabular}{c|c}
		\multirow{2}{*}{\textbf{Aanwezig}} & Tim Witters, Youri Coppens, Pieter Meiresone, \\
		& Nicolas Carraggi,  Christophe Gaethofs, Fernando Suarez\\
		\hline
		\textbf{Afwezig} & \\
	\end{tabular}
\end{table}

\subsection{Verantwoordelijkheden}
\begin{itemize}
	\item Voorzitter: Pieter Meiresone
	\item Secretaris: Youri Coppens
\end{itemize}

\subsection{Vooruitgang sinds vorige vergadering}
\subsection{Vergaderingspunten}
Tijdens deze vergadering dienen volgende vergaderingspunten besproken te worden.
\begin{table} [H]
	\centering
	\begin{tabular} {l|c|c|c}
		Onderwerp & Tijdsduur & Prioriteit & Beschrijving \\
		\hline
		Verslag Vergadering 5 & 2min & Hoog & Door Fernando \\
		Time Tracking & 2min & Hoog & \\
		Leden Pagina & 2min & Laag & \\
		Documenten Template & 2min & Middelmatig & Logo ontbreekt nog op de documenten \\
		Workshop Christophe & 45min & Middelmatig & \\
		Brainstorm & 15min & Middelmatig & Project Risico's \\
		SPMP & 2u & Hoog & Feedback overlopen Ragnild \\
		SRS & 2u & Hoog & Requirements groeperen (Tim) \\
		SRS & 2u & Hoog & Stappenplan voor requirements (Youri) \\
		SRS & 2u & Middelmatig & Screenshots GUI in Photoshop (Fernando) \\
		SCMP & 2u & Hoog & Procedures uitwerken \\
	\end{tabular}
\end{table}

Actiepuntjes alvorens we effectief starten.
\begin{itemize}
\item ! Fernando moet de verslagen aanvullen (vergadering 5) en timetracking doen.
\item ! Youri: Templates documenten veranderen: logo calzone verwerken.
\end{itemize}

\begin{itemize}
\item Christophe geeft een inlichtingssessie/workshop over Git. 
\item ! Iedereen gebruikt SourceTree als Git GUI client
\item Pieter wil dat de website meer integreert met de github om de verslagen te tonen ipv deze telkens naar Wilma te ssh'en
\item Christophe toont de website en meldt dat we bij de timetracking beter beschrijven en namen geven aan onze topics (bvb vergaring 1,2,3 ipv telkens het woord vergadering)
\item Next topic: risico's brainstorm. Wat hebben we tot nu toe bedacht:
	\begin{itemize}
	\item niet-realistische planning
	\item geen ervaring
	\item performatie scheduler
	\item backend faalt: wilma die uitvalt
	\item verkeerde functionaliteit
	\item cross browser support
	\item merge conflicts
		\begin{itemize}
		\item 4 onderverdelinge: tussen klein/groot en functioneel/niet-functioneel
		\item klein meningsverschil: door spmp
		\item groot meningsverschil: door meeting en daarna stemming
		\end{itemize}
	\item requirements die veranderen
	\item communicatie die misloopt
	\item iemand die ziek valt
	\item Interne afspraken worden niet nageleefd
	\item meningsverschillen
	\item spellingsfouten
	\item slechte documentatie bij code; Oplossing: JavaDoc, richtlijnen voor documentatie
	\item bugs > zie problem resolution plan
	\end{itemize}
\item En nu gaan we arbeiten:
	\begin{itemize}
	\item Tim: SRS herverdeling files
	\item Fernando: photoshoppen
	\item Youri: stappenplannen van de bemangrijkste requirements uitschrijven
	\item de rest: SPMP verbeteren adhv feedbackpunten
	\end{itemize}
\end{itemize}

\subsection{Action items tegen volgende vergadering}
\subsection{Agenda Punten volgende vergadering}

