\section{Vergadering 26/11/2013}
\subsection{Aanwezigheidslijst}
\begin{table}[htbp]
	\centering
	\begin{tabular}{c|c}
		\multirow{2}{*}{\textbf{Aanwezig}} & Tim Witters, Youri Coppens, Pieter Meiresone, \\
		& Nicolas Carraggi,  Christophe Gaethofs, Fernando Suarez, Sam van den Vonder \\
		\hline
		\textbf{Afwezig} & \\
	\end{tabular}
\end{table}

\subsection{Vooruitgang sinds vorige vergadering}


\subsection{Verantwoordelijkheden}
\begin{itemize}
	\item Voorzitter: Pieter Meiresone
	\item Secretaris: Tim Witters
\end{itemize}
\subsection{Vergaderingspunten}

Tijdens deze vergadering dienen volgende vergaderingspunten besproken te worden.
\begin{table} [H]
	\centering
	\begin{tabular} {l|cll}
		\textbf{Onderwerp} & \textbf{Tijdsduur} & \textbf{Prioriteit} & \textbf{Beschrijving} \\ % Prioriteit: Hoog, Middelmatig of Laag
		\hline
		Progressie & 5min & Hoog & Progressie sinds vorige vergadering bespreken \\

	\end{tabular}
\end{table}

Andere punten die vermeld dienen te worden:

\begin{itemize}
	\item
	{
		Rechten op GitHub is een probleem:
		\begin{itemize}
			\item Enkel de verantwoordelijken kunnen milestones aanmaken
			\item Bij het submitten van een nieuw issue kan nieuw toegewezen worden
			\item Issues kunnen niet meer gesloten worden.
		\end{itemize}	
	}
\end{itemize}

\begin{itemize}
	\item Vanaf nu doet Tim Witters vergaderen rechtstreeks in latex
	\item Communicatie moet beter verlopen 
		\begin{itemize}
			\item Vanaf als er een probleem is of iets niet duidelijk
			\item
		\end{itemize}
	\item Milestones overlopen en approved
		\begin{itemize}
			\item Goedgekeurd door de aanwezigen
		\end{itemize}
	\item ORM overlopen usermanagement
		\begin{itemize}
			\item 1 of 2 soorten flags voor gebruikerstype? Keuze kan gemaakt worden tijdens implementatie
			\item Uren bij datum class?
			\item UML moet nog gedaan worden bij design
		\end{itemize}
	\item Gebruik van bootstrap voor webstie. bootswatch.com/yeti template zal gebruikt worden. Gebruik van standaard kleuren van deze template. De mock-ups worden gebruikt als referentie voor het ontwerp. User interfaces in SRS zullen niet meer worden aangepast en later pas worden vervangen door screenshots.
	\item SPMP beknopt weergeven in zal gewoon manueel gebeuren en we maken geen gebruik van MS Project. Dit omdat het te ingewikkeld is met variabele uren in onze situatie.
	\item Indelen team
\end{itemize}



\subsection{Action items tegen volgende vergadering}
\begin{table} [H]
	\centering
	\begin{tabular} {l|l|l}
		\textbf{Activiteit} & \textbf{Team} & \textbf{Opmerkingen} \\
		\hline
		Implementation onderzoek & Tim & \\
		Infrastructuur & Sam,Pieter, Chritophe & \\
		Design (UML/ORM) & Youri, Nikolas, Fernando & \\
	\end{tabular}
\end{table}

\subsection{Agenda Punten volgende vergadering}
	\begin{itemize}
		\item Komende donderdag van 12-13u
		\item Verdeling teams coding
	\end{itemize}


