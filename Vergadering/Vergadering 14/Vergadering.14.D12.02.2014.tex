\section{Vergadering \MeetingDate}
\subsection{Praktische informatie}
\subsubsection{Tijd \& Plaats}
Uur: 10:00 - 13:00
\\
Lokaal: E0.02

\subsubsection{Aanwezigheidslijst}
\begin{table}[htbp]
	\centering
	\begin{tabular}{c|c}
		\multirow{2}{*}{\textbf{Aanwezig}} & Tim Witters, Youri Coppens, Pieter Meiresone, \\
		& Nicolas Carraggi,  Christophe Gaethofs, Fernando Suarez, Sam van den Vonder \\
		\hline
		\textbf{Afwezig} & \\
	\end{tabular}
\end{table}


\subsubsection{Verantwoordelijkheden}
\begin{itemize}
	\item Voorzitter: Pieter Meiresone
	\item Secretaris: Tim Witters
\end{itemize}


\subsubsection{Kritiekpunten op verslag vorige vergadering}
Éen enkele opmerking over de maximale paswoord lengte. Tijdens het development is er besloten om de maximale paswoord lengte vast te leggen op 64 karakters. De reden voor deze maximale paswoord lengte is voornamelijk om praktische redenen: een paswoord met lengte groter dan 64 is moeilijk te onthouden.

\subsection{Vooruitgang sinds vorige vergadering} \label{sec:Vooruitgang}
Geen vooruitgang.

\subsection{Vergaderingspunten} 
Tijdens deze vergadering dienen volgende vergaderingspunten besproken te worden.
\begin{table} [H]
	\centering
	\begin{tabular} {l|c|c|c}
		Onderwerp & Tijdsduur & Prioriteit & Beschrijving \\ % Prioriteit: Hoog, Middelmatig of Laag
		\hline
		Evaluatie & 30min & Hoog & / \\
		Bespreking Feedback & 30min & Hoog & / \\
		Planning Volgende Iteraties & 30min & Hoog & / \\
		
	\end{tabular}
\end{table}

\subsubsection{Evaluatie}
Een evaluatie van de verschillende teamleden en de groep in zijn geheel is nuttig om zwakke punten in de groep bloot te leggen en hier aan te werken. De zwakke punten in de groep waren voornamelijk time-management en samenwerking.
\\
\\
Vorig semester had iedereen een druk lessenrooster, dit zorgde ervoor dat samenkomen en progressie soms niet liepen zoals verwacht. Vermits komend semester voor iedereen rustiger is, verwachten we hier minder problemen mee.
\\
\\
Op gebied van samenwerking is alles ook niet vlekkenloos verlopen, zonder in details te treden kunnen we zeggen dat we hierover samengezeten, meningsverschillen opgelost alsook lessen uitgeleerd hebben. Vermits de groep nu meer één geheel is, verwachten we een betere samenwerking.
\\
\\
Specifieke problemen met bepaalde teamleden waren er niet.
\\
\\
Algemeen zijn we het er in de groep over eens dat komend semester goed zal verlopen.

\subsubsection{Bespreking Feedback}
\paragraph{SRS}
\subparagraph{Feedback:}
Sectie 3.1.4: Is nog altijd een beetje te uitgebreid qua functionaliteit.
\\
Sectie 3.1.7: De screenshots zijn al een goed begin, maar nog niet echt nuttig zonder schematisch overzicht en zonder link naar de requirements.
\\
Algemeen: Blijft een goed document, maar de gevraagde aanpassingen zijn niet gebeurd.
\subparagraph{Bespreking:}

\begin{itemize}
	\item Sectie 3.1.4 zullen 2 personen bij elkaar komen om dit op te lossen
	\item Alle foto's van user interfaces zullen bij respectievelijke requirements gezet worden. 
	\item Er moet nog worden nagedacht over extra functionaliteit ivm mobiele smartphones. Dit issue is aangemaakt op github issue \#68 in CalZoneVUB/Documents.
\end{itemize}


\paragraph{STD}
\subparagraph{Feedback:}
Degelijke eerste versie.
Leg duidelijk(er) uit wat er gebeurt als bepaalde testen (integratie, verificatie) falen: wie rapporteert, wie volgt dit op, en wie is verantwoordelijk voor het oplossen van problemen?

\subparagraph{Bespreking:}
Aanpassen van punten aangegeven hierboven. En onderzoeken hoe unit test worden afgehandeld op de website: voornamelijk weergave en updates.

\paragraph{SDD}
\subparagraph{Feedback:}
Sectie 2.1 Model: Goed dat jullie dit patroon (met figuur) opnemen, maar dit is niet het eerste en belangrijkste.
De algemene systeemarchitectuur en het high-level design zijn belangrijker.
\\
Sectie 3.3 Data: Leest erg moeilijk (vooral i.v.m. activatiesleutels). Waarom is er een lijst?
Wordt de sessie in de database opgeslagen? Motiveer deze keuze.
Wordt er nergens anders dan op de backend data bijgehouden (bvb. frontend+caching)?
\\Éen enkele opmerking over de maximale paswoord lengte. Tijdens het development is er besloten om de maximale paswoord lengte vast te leggen op 64 karakters. De reden voor deze maximale paswoord lengte is voornamelijk om praktische redenen: een paswoord met lengte groter dan 64 is moeilijk te onthouden.
Algemeen: Redelijke eerste versie. Zorg wel voor een betere beschrijving (met schema's) van de architectuur en het high-level design.
Leg ook duidelijk uit hoe informatie door het systeem vloeit (richting, formaat, ...).
Verder moeten de meeste secties nog uigebreid worden.
\subparagraph{Bespreking:}
Youri en Christophe zullen hierbij samenzitten om deze punten aan te pakken.
 

\subsubsection{Planning Volgende Iteraties}
\begin{itemize}
	\item
	{
		Tweede Iteratie
		\begin{itemize}
			\item FR 1: User Management
			\item FR 2: Vakken
			\item FR 3: Lokalen
			\item NFR 1: Beveiliging
			\item Documenten afwerken
		\end{itemize}
	}
	\item
	{
		Derde Iteratie
		\begin{itemize}
			\item FR 4: Schedular
			\item FR 5: Lessenroosters bekijken
			\item FR 6: Functionaliteit voor de programmabeheerder
			\item Mobiele Functionaliteit
		\end{itemize}
	}
	\item
	{
		Vierde Iteratie
		\begin{itemize}
			\item Afwerken vorige onderdelen
		\end{itemize}
	}
\end{itemize}

\subsection{Volgende vergadering}
De volgende vergadering vindt plaats op vrijdag 21/02/2014 13u, verzamelen aan IG.
Volgende vergadering vind dan plaats op disdag 25/02/2014.
\subsubsection{Activiteiten tegen volgende vergadering} \label{sec:TODOActiviteiten}
\begin{table} [H]
	\centering
	\begin{tabular} {l|l|l}
		\textbf{Activiteit} & \textbf{Team} & \textbf{Opmerkingen} \\
		\hline
		 STD aanpassingen & Sam  & \\
		 SDD aanpassingen & Youri, Chritophe & \\
		 SRS aanpassingen & Fernando & \\
		 Verdeling werk implementatie & Tim & \\
		 Datadump analyzeren & Nicolas & \\ 
		 SPMP afwerken & Pieter & \\
	\end{tabular}
\end{table}

\subsubsection{Agenda Punten volgende vergadering}
\begin{itemize}
	\item Bespreking documenten
	\item Verdeling werk implementatie
\end{itemize}