\section{Vergadering \MeetingDate}
\subsection{Praktische informatie}
\subsubsection{Tijd \& Plaats}
Uur: 10:00 - 13:00
\\
Lokaal: E0.04
\subsubsection{Aanwezigheidslijst}
\begin{table}[htbp]
	\centering
	\begin{tabular}{c|c}
		\multirow{2}{*}{\textbf{Aanwezig}} & Tim Witters, Youri Coppens, Pieter Meiresone, \\
		& Nicolas Carraggi,  Christophe Gaethofs, Fernando Suarez, Sam van den Vonder \\
		\hline
		\textbf{Afwezig} & \\
	\end{tabular}
\end{table}

\subsubsection{Verantwoordelijkheden}
\begin{itemize}
	\item Voorzitter: Pieter Meiresone
	\item Secretaris: Tim Witters
\end{itemize}

\subsubsection{Kritiekpunten op verslag vorige vergadering}
Niet van toepassing, eerste vergadering van het tweede semester.

\subsection{Vooruitgang sinds vorige vergadering} \label{sec:Vooruitgang}
Geen vooruitgang.

\subsection{Vergaderingspunten} 
Tijdens deze vergadering dienen volgende vergaderingspunten besproken te worden.
\begin{table} [H]
	\centering
	\begin{tabular} {l|c|c|c}
		Onderwerp & Tijdsduur & Prioriteit & Beschrijving \\ % Prioriteit: Hoog, Middelmatig of Laag
		\hline
		Evaluatie & 30min & Hoog & \\
		Bespreking Feedback & 30min & Hoog & \\
		Planning Volgende Iteraties & 30min & Hoog & \\
		
	\end{tabular}
\end{table}

\subsubsection{Evaluatie}
Een evaluatie van de verschillende teamleden is nuttig om zwakke punten in de groep bloot te leggen en hier aan te werken.

\begin{table} [H]
	\centering
	\begin{tabular} {l|c}
		 & Bespreking \\
		\hline
		Algemeen & \\
		Christophe & \\
		Fernando & \\
		Nicola & \\				
		Pieter & \\
		Sam & \\		
		Tim & \\
		Youri & \\
	\end{tabular}
\end{table}

\subsubsection{Bespreking Feedback}
\paragraph{SRS}
\subparagraph{Feedback:}
Sectie 3.1.4: Is nog altijd een beetje te uitgebreid qua functionaliteit.
\\
Sectie 3.1.7: De screenshots zijn al een goed begin, maar nog niet echt nuttig zonder schematisch overzicht en zonder link naar de requirements.
\\
Algemeen: Blijft een goed document, maar de gevraagde aanpassingen zijn niet gebeurd.
\subparagraph{Bespreking:}

\paragraph{STD}
\subparagraph{Feedback:}
Degelijke eerste versie.
Leg duidelijk(er) uit wat er gebeurt als bepaalde testen (integratie, verificatie) falen: wie rapporteert, wie volgt dit op, en wie is verantwoordelijk voor het oplossen van problemen?

\subparagraph{Bespreking:}

\paragraph{SDD}
\subparagraph{Feedback:}
Sectie 2.1 Model: Goed dat jullie dit patroon (met figuur) opnemen, maar dit is niet het eerste en belangrijkste.
De algemene systeemarchitectuur en het high-level design zijn belangrijker.
\\
Sectie 3.3 Data: Leest erg moeilijk (vooral i.v.m. activatiesleutels). Waarom is er een lijst?
Wordt de sessie in de database opgeslagen? Motiveer deze keuze.
Wordt er nergens anders dan op de backend data bijgehouden (bvb. frontend+caching)?
\\
Algemeen: Redelijke eerste versie. Zorg wel voor een betere beschrijving (met schema's) van de architectuur en het high-level design.
Leg ook duidelijk uit hoe informatie door het systeem vloeit (richting, formaat, ...).
Verder moeten de meeste secties nog uigebreid worden.
\subparagraph{Bespreking:}

\subsubsection{Planning Volgende Iteraties}
\begin{itemize}
	\item
	{
		Tweede Iteratie
		\begin{itemize}
			\item FR 1: User Management
			\item FR 2: Vakken
			\item FR 3: Lokalen
			\item NFR 1: Beveiliging
		\end{itemize}
	}
	\item
	{
		Derde Iteratie
		\begin{itemize}
			\item Schedular
		\end{itemize}
	}
	\item
	{
		Vierde Iteratie
		\begin{itemize}
			\item Blabla
		\end{itemize}
	}
\end{itemize}

\subsection{Volgende vergadering}
De volgende vergadering vindt plaats op dinsdag 13/12/2013 18u, verzamelen aan IG.
\subsubsection{Activiteiten tegen volgende vergadering} \label{sec:TODOActiviteiten}
\begin{table} [H]
	\centering
	\begin{tabular} {l|l|l}
		\textbf{Activiteit} & \textbf{Team} & \textbf{Opmerkingen} \\
		\hline
		 &  & \\

	\end{tabular}
\end{table}

\subsubsection{Agenda Punten volgende vergadering}
