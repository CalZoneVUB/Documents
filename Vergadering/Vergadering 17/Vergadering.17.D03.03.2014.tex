\section{Vergadering \MeetingDate}
\subsection{Praktische informatie}
\subsubsection{Tijd \& Plaats}
Uur: 16:00 - 18:00
\\
Lokaal: E0.03
\subsubsection{Aanwezigheidslijst}
\begin{table}[htbp]
	\centering
	\begin{tabular}{c|c}
		\multirow{2}{*}{\textbf{Aanwezig}} & Tim Witters, Youri Coppens, Pieter Meiresone, \\
		& Nicolas Carraggi,  Christophe Gaethofs, Fernando Suarez, Sam van den Vonder \\
		\hline
		\textbf{Afwezig} & \\
	\end{tabular}
\end{table}

\subsubsection{Verantwoordelijkheden}
\begin{itemize}
	\item Voorzitter: Pieter Meiresone
	\item Secretaris: Tim Witters
\end{itemize}

\subsubsection{Kritiekpunten op verslag vorige vergadering}
Er is kritiek op de vooropgestelde werkwijze door de implementation manager Tim en project manager Tim: deze stelden voor dat er een duidelijke scheiding komt tussen Front End, Back end en connectie daartussen. De krtiiek bestaat uit het feit dat Sam vindt dat elk teamlid bij elk aspect van het project even sterk betrokken moet zijn. 
\\
\\
Vermits we met veel teamleden zijn, is er eigenlijk geen alternatief en moeten we wel degelijk verschillende gebieden goed afbakenen. Er is dan ook besloten met deze werkwijze verder te gaan.
\subsection{Vooruitgang sinds vorige vergadering}
\begin{itemize}
	\item Revisie van algemene design: voornamelijk homepagina, kalenderweergave en huisstijl.
	\item Front end voor FR 2.5, FR 2.6 en FR 2.7
	\item Datadump parsen
	\item Progressie aan database.
\end{itemize}

\subsection{Vergaderingspunten}
Tijdens deze vergadering dienen volgende vergaderingspunten besproken te worden.
\begin{table} [H]
	\centering
	\begin{tabular} {l|c|c|c}
		Onderwerp & Tijdsduur & Prioriteit & Beschrijving \\ % Prioriteit: Hoog, Middelmatig of Laag
		\hline
		Progressie bespreken & 1u & Hoog & Elkaar inlichten \\
		Brainstormen & 1u & Middelmatig & Gezamenlijk brainstormen \\
	\end{tabular}
\end{table}

\subsection{Volgende vergadering}
Nog te bepalen.
\subsubsection{Activiteiten tegen volgende vergadering} \label{sec:TODOActiviteiten}
\begin{table} [H]
	\centering
	\begin{tabular} {l|l|l}
		\textbf{Activiteit} & \textbf{Team} & \textbf{Opmerkingen} \\
		\hline
		FR 2.5, FR 2.6 en FR 2.7 & Sam, Fernando & Layout in orde brengen, alle knoppen toevoegen, \\
		& & logic in de controllers, GEEN validators \\
		FR 3.1, FR 3.2 en FR 3.3 & Sam, Fernando & Zelfde opmerking. \\
		Navigatiebalk in orde maken & Fernando & Linksboven staat er een pijltje. Als de gebruiker \\
		& & aangemeld is, moeten er extra velden zichtbaar zijn. \\
		Security per pagina & Pieter & M.b.v. annotaties -$>$ opzoeken \\
	\end{tabular}
\end{table}
\textbf{Opmerking:} een belangrijk onderscheid dat gemaakt moet worden bij vakken is dat professors vakken moeten kunnen aanmaken/abonneren, terwijl dat studenten zich enkel kunnen abonneren. Zorg ervoor dat hier zeker de juiste knoppen voor op voorzien op de webpagina.
\\
\\
Probeer liefst zoveel mogelijk af heb te hebben voor donderdagmiddag 6 maart. Ik en Tim zitten 's avonds al samen om het geheel af te werken. Degene die willen/kunnen mogen ons vergezellen!

\subsubsection{Agenda Punten volgende vergadering}
Nog te bepalen.