\appendix
\chapter{Metrieken} \label{chap:Metrics}
In paragraaf \ref{sec:Metrics} is er reeds een kort overzicht gegeven van de metrieken die we zullen verzamelen in de loop van dit project. In dit hoofdstuk staat een meer gedetailleerde beschrijving. Op methodenniveau bestuderen we volgende metrieken:
\begin{enumerate}
	\item 
		Cyclomatic Complexity.
		\\
		\\
		Deze metriek geeft een indicatie van het aantal `lineaire' segementen in een methode (m.a.w. stukken code met geen branches). Dit kunnen we onder andere gebruiken om het aantal tests te bepalen om volledige dekking te krijgen. Het geeft ook een indicatie van de complexiteit van de methode.
		
	\item 
		Aantal statements
		\\
		\\
		Om de grootte van de methode te onderhouden, maken we gebruik van het aantal statements binnen een methode. Deze metriek is robuuster dan het aantal lijnen code vermits deze onafhankelijk is van de gebruikte programmeerstijl.
	\item
		Aantal levels.
		\\
		\\
		Deze metriek geeft het maximaal aantal geneste lagen in een methode. Een hoog aantal levels geeft aan dat we te maken hebben met een complexe  methode. Deze methoden kunnen vereenvoudigd worden door het extraheren van verscheidene private methoden.
	
	\item
		Aantal lokale variabelen in de scope.
		\\
		\\
		Deze metriek geeft het maximaal aantal lokale variable dat zich in de scope bevindt gedurende elk mogelijk punt in de methode. Een groot aantal wijst op complexe methoden.
	\item
		Aantal parameters.
		\\
		\\
		Deze metriek bevat het aantal parameters dat doorgegeven wordt aan een methode. Een te hoog aantal parameters wijst erop dat er te weinig gebruik gemaakt wordt van klassen.
	
	\item 
		Feature envy
		\\
		\\
		Deze metriek geeft aan in welke mate de methode ge\"{i}terreseerd is in methoden en attributen van andere klassen. Wanneer deze metriek een hoge waarde heeft, is het beter de methode te verplaatsen naar de klasse waarvan het het meest gebruik maakt. Wanneer dit gedrag slechts deels door een methode wordt bepaald, is het aangewezen om dit deel van de methode te verplaatsen (indien mogelijk).
		\\
		\\
		Voor de berekening gaan we als volgt te werk. Zij $m$ de methode waarvoor we de ``feature envy'' willen berekenen. We nemen dan $F_{c}$ de verzameling van features dat gebruikt worden door $m$ in klasse $c$. Verder is $c_{m}$ de klasse in welke de methode $m$ gedefinieerd is. We defini\"{e}ren dan de feature envy als:
		$$ FE = \max_{c \neq c_{m}} (|F_{c}|) - |F_{c_{m}}| $$
		Wanneer $FE > 0$ betekent dat de methode $m$ meer features gebruikt van een andere klasse. De klassenstructuur moet dan herzien worden. 

\end{enumerate}
Op klassenniveau verzamelen we volgende metrieken:
\begin{enumerate}
	\item 
		Efferent Couplings.
		\\
		\\
		In deze metriek wordt er gemeten hoeveel andere klassen de huidige klasse kent. Een hoog aantal koppelingen is nadelig voor de betrouwbaarheid van de code vermits het afhankelijk is van verscheidene types. Door de klasse op te delen in verscheidene deelklassen, kunnen we het aantal koppelingen naar omlaag brengen.
	\item
		Aantal attributen.
		\\
		\\
		Deze metriek meet het aantal attributen in een klasse. Bij een groot aantal attributen moet er nagegaan worden of er verscheidene attributen kunnen gegroepeerd worden in deelklassen.
	
	\item
		Complexiteit
		\\
		\\
		Deze metriek wordt berekend door de som te nemen van de Cyclomatic Complexities van de verschillende methoden die zich in deze klasse bevinden. Deze stelt dus de complexiteit voor van de gehele klasse.
		
	\item 
	{
		Cohesie tussen de verschillende methodes.
		\\
		\\
		De cohesie tussen de verschillende methoden in \'{e}\'{e}n enkele klasse is een belangrijk concept bij object georienteerd programmeren. De cohesie geeft aan of de klasse \'{e}\'{e}n of meerdere abstracties voorstelt. Indien \'{e}\'{e}n klasse meerdere abstracties voorstelt, moet deze opggesplitst worden in meerdere klassen waarbij elke klassen \'{e}\'{e}n abstractie voorstelt.
		\\
		\\
		Wij zullen de Henderson-Sellers metriek gebruiken \cite{HendersonSellers}. Om deze te berekenen voeren we volgende variabelen in: 
		$$ M = \{ m | \text{m is methode van de klasse} \} $$
		$$ F = \{ f | \text{f is een veld van de klasse} \} $$
		$$ r : F \rightarrow \mathbb{N} $$
Hierbij berekent $r(f)$ het aantal methoden dat veld $f$ aanroept. Vervolgens defini\"{e}ren we ook $\bar{r}$ als het gemiddelde van $r(f)$ over $F$. De Henderson-Seller metriek wordt dan berekent als:
		$$ HS = \frac{\bar{r} - |M|}{1 - |M|} $$
		Hoe lager de waarde, hoe beter de cohese tussen de verschillende methoden.
	} %TODO eventueel referentiewaarde en verklaren vanuit formule
\end{enumerate}