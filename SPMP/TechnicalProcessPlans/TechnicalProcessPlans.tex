\chapter{Technisch process plan}
\section{Process model}
Vanwege de opgelegde deadlines (zie tabel \ref{tab:kalender}), zullen we gebruik maken van een iteratief model voor het opleveren van de documenten en code. Hierbij zal er steeds geittereerd worden over requirements analyse, design, constructie, testing en installatie.
%Eventueel foto.
\section{Methodes, tools and technieken} \label{sec:languages}
Voor dit project zullen we enkel gebruik maken van Java, JavaScript, HTML, CSS en SQL als programmeertaal. Andere bijhorende open-source frameworks en bibliotheken kunnen ook gebruikt worden. Voor testen te schrijven zullen we gebruik maken van het JUnit framework. Dit alles zal gebeuren in Eclipse. 
\\
\\
Er zal gebruik worden gemaakt van een public repository op GitHub \cite{GitHubRepository} voor het verzamelen van de code. Verder zal er ook een repository voorzien zijn voor de verschillende documenten. Deze repositories zijn ook gekoppeld met onze website.

\section{Infrastructuur plan}
Voor dit project zullen we gebruik maken van de wilma server van de Vrije Universiteit Brussel \cite{WilmaServer}. Deze server bevat een mysql database. Verder wordt er gebruik gemaakt van het netwerk van de VUB.

% Opmerking: Product acceptance plan niet