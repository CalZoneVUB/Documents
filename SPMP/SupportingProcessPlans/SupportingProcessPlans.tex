\chapter{Supporting process plans}
\section{Configuration management plan} \label{SoftwareConfigurationManagementPlan}
In dit onderdeel van het SPMP wordt het Software Configuration Management Plan, of kortweg SCMP, kort besproken. Een apart document voor het SCMP is voor dit project overbodig aangezien meerdere onderwerpen reeds uitvoerig in andere delen van dit document aan bod komen. Evidente onderdelen van het SCMP, zoals een beschrijving van het software project, zullen dan ook worden weggelaten.

\subsection{Introductie}
Het Software Configuration Management Plan heeft als doel een gestructureerd overzicht te cre\"{e}ren waarin wordt beschreven op welke manier er met de software wordt omgegaan, hoe deze wordt gebruikt, hoe deze in gebruik wordt gecontroleerd en hoe de werking van het team wordt gestuurd binnen bepaalde gebruiksnormen.
\\
\\
Dit onderdeel van dit document is bedoeld als richtlijn voor het gebruik van de software binnen het project en is gericht aan de teamleden die meewerken aan dit project, aan de Configuration Manager die deze richtlijnen dient te implementeren, te controleren en dient in te grijpen indien nodig, alsook aan derden die op deze manier de structuur en interne configuratie van werken kunnen volgen.
\\
\\
Omdat het gebruik van software en/of systemen stap per stap wordt geadopteerd, zal een betere vertrouwdheid hiermee een duidelijker beeld vormen over de voor- en nadelen. Het is dan ook vanzelfsprekend dat dit document, alsook de hierin beschreven systemen en software, mee zal evolueren naarmate dit nodig zal zijn. Dit telkens met oog op de vergemakkelijking van de samenwerking, verbetering van de communicatie en de verhoging van de productiviteit.

\subsection{SCM Management en verantwoordelijkheden}
Dit deel van het document beschrijft de allocatie van de verantwoordelijkheden en machtigingen voor de verscheidene SCM activiteiten, en het beheer hiervan.
\\
\\
De in dit onderdeel beschreven richtlijnen voor het gebruik van de aangeboden en verworven tools, zijn van toepassing voor elk teamlid. Het is de verantwoordelijkheid van deze leden om zich hiermee vertrouwd te maken en deze toe te passen binnen dit project. Met problemen of vragen over de gebruikte software, kunnen zij terecht bij de Configuration Manager: Christophe Gaethofs.
\\
\\
Het is de taak van de Configuration Manager, Christophe Gaethofs, om hulp te verlenen aan de andere teamleden, toezicht te houden dat de teamleden deze richtlijnen volgen, alsook hen op de hoogte te brengen over veranderen of bijsturingen gemaakt aan het SCMP, hetzij door een collectief of individueel genomen besluit. Slechts na unanieme goedkeuring, zullen bijsturingen effectief worden geïmplementeerd en opgenomen in het SCMP.
\\
\\
Verder zal de Configuration Manager verantwoordelijk zijn voor de configuratie en controle van al de in sectie \ref{sec:SCMActiviteiten} opgenomen activiteiten.
\\
\\
Voor deze opdracht zal de Configuration Manager, Christophe Gaethofs, bijgestaan worden door Tim Witters, die de taak van Configuration Manager op zich neemt bij afwezigheid van Christophe Gaethofs, of waneer de situatie dit vereist. Dit op voorwaarde dat deze hiervan op tijd op de hoogte wordt gebracht, om hem de mogelijkheid te geven zich voor eventuele taken in te werken.

\subsection{SCM Activiteiten} \label{sec:SCMActiviteiten}
Dit deel van het document identificeert alle functies en taken die nodig zijn om de configuratie van het beschreven software project en bijhorende tools te beheren. Hierbij wordt rekening gehouden met de bij het project opgelegde procedures voor het indienen en beheren. Voor de controle, het beheer en het overzicht van alle activiteiten, zal voornamelijk de team website \cite{portalWebsite} en GitHub gebruikt worden.
\\
\\
Alle gebruikte en in dit document aangehaalde tools, die een API ter beschikking hebben en bijdragen aan de workflow, zullen in deze website worden geïntegreerd.

\subsubsection{Website}
De team website zal intensief gebruikt worden als platform dat voor alle leden ter beschikken wordt gesteld ter collaboratie. Het zal onder meer worden gebruikt om een overzicht te cre\"{e}eren voor:
\begin{itemize}
 \item updates, gemaakt aan het SRS of updates aan andere documenten;
 \item gedetailleerde informatie van alle activiteiten, beschreven in het Project Plan;
 \item opvolging van de vooruitgang (o.a. vooruitgang van de requirements);
 \item opvolging van de besteedde tijd per activiteit per lid via Time Tracking;
 \item de communicatie door correspondentie op te lijsten in een overzicht;
 \item code-overzicht, door volledige integratie van GitHub. 
\end{itemize}

De website wordt ook gebruikt als tool om een eenvoudig inzicht te geven in de individuele bijdrages door de leden, aan de hand van alle ingevoerde gegevens. 
\\
\\
Hoewel dit in eerste instantie meer werk met zich meebrengt, zal dit werk zich snel vertalingen in een betere samenwerking waarbij communicatie, duidelijkheid en overzicht primeert. Zo geeft het de teamleider en andere verantwoordelijken gecentraliseerde toegang tot alle informatie om eventuele bijsturingen van teamleden, manier van werken, programma’s, ... snel te kunnen laten gebeuren.
\\
\\
De implementatie, onderhoud en controle van de website is de verantwoordelijkheid van de Webmaster en Configuration Manager. Beide taken zullen worden vervuld door Christophe Gaethofs. Bij eventuele conflicten of problemen op de website, dienen de leden deze hiervan dan ook zo snel mogelijk op de hoogte te brengen om deze problemen te kunnen oplossen.
\\
\\
De communicatie op de website zal verlopen via een berichtenpagina, waarop leden mededelingen en berichten kunnen posten. Dit systeem werkt als een blog waarbij het gepubliceerde artikel, in deze context dan bericht, direct wordt weergegeven op de website. Dit bericht wordt, vanaf het adres van het lid dat deze publicatie maakte, ook als e-mail verstuurd naar de projectmailbox. Alle leden krijgen dit bericht dan in hun inbox, maar kunnen dit ook online bekijken. De communicatiepagina's stellen hen ook in staat online te reageren op deze berichten via Disqus (\url{http://disqus.com/}). Deze open source tool geeft een website de mogelijkheid om een reactieformulier aan een pagina toe te voegen en dit te beheren. Elke berichtpagina zal onder het bericht zo'n formulier bevatten. De geplaatste reacties worden niet via mail verstuurd.
\\
\\
De website is gebouwd op het opensource CMS (Content Management System) SilverStripe (\url{http://www.silverstripe.org}) en maakt gebruikt van het Bootstrap JS (\url{http://getbootstrap.com/}) framework voor het thema en JavaScript functionaliteit. Voor de kalender is een eigen module ontworpen op basis van Bootstrap Calendar (\url{http://bootstrap-calendar.azurewebsites.net}), een opensource kalender voor Bootstrap JS.
\\
\\

\subsubsection{Documenten en source code}
Om alle fasen van het project ordelijk en gestructureerd te kunnen laten verlopen, zal er gebruikt gemaakt worden van publieke repositories, aangeboden door GitHub.
\\
\\
De configuratie hiervan dient in lijn te zijn met de opgegeven voorwaarden waardoor alle documenten en source code overeenkomstig zullen worden getagd/gebranchd (se2-iterN waarbij N staat voor het iteratienummer) in de repository.
\\
\\
Voor de documentatie en de source code worden verschillende repositories worden aangemaakt.
\\
\\
De in de repository ondergebrachte documenten, worden in Latex geschreven en bijgehouden in branches op de Git. Hierdoor zijn de stabiele documenten voor alle leden beschikbaar, worden alle wijzigingen en versies bijgehouden en kan er simultaan aan hetzelfde document worden gewerkt zonder conflicten.
\\
\\
Om problemen bij het mergen van branches te vermijden en controle te houden op de gemaakte wijzigingen op de Git, zal voor het coderen gebruikt worden gemaakt van forks. Dit zijn clones van de eigenlijke team repository in de persoonlijke repositories van de leden. Hierin kan men verder werken zonder conflicten te veroorzaken. Als een lid  wijzigingen wil doorvoeren, dient deze een pull request uit te voeren naar de team repository. De Configuration Manager, Christophe Gaethofs, zal dan samen met de reserve Configuration Manager, Tim Witters, al deze pull requests behandelen en doorvoeren indien deze geen conflicten veroorzaken.
\\
\\
Alle source code is terug te vinden in de Git, toegankelijk via de website. Als editor zal er gebruikt worden gemaakt van twee open source git editors. Voor een eenvoudige werking zonder te veel verwarrende functionaliteit, zal gebruik gemaakt worden van een GitHub client. Voor de meer geavanceerde functies en het beheer van de repositories, zullen de Configuration Managers, gebruik maken van SourceTree \url{http://www.sourcetreeapp.com}.
\\
\\
Er wordt nooit online in de Git zelf gewerkt, tenzij bepaalde goed gegronde redenen lokaal werken niet mogelijk maken en andere teamleden hiermee akkoord gaan.
\\
\\
In elke fase van het project wordt er ook onderling besproken hoe we het eventuele branchen van een repository gaan aanpakken bij het opdelen van de implementatie-taken.

\subsubsection{Issues \& Milestones}
De melding en opvolging van problemen gebeurt ook via GitHub. Dit stelt ons in staat om code-specifieke issues te melden en hiervoor een verantwoordelijke aan te duiden. Via de comments kan er gereageerd worden en kan een probleem worden gesloten wanneer het is opgelost. Dit laatste wordt ook opgevolgd door de Configuration Manager die, buiten de orde en netheid van de repositories te bewaren, ook toezicht houdt over alle open problemen of bugs. Ook voor problemen of opmerkingen bij de documentatie zal hier van gebruik worden gemaakt.
\\
\\
Problemen die geen betrekking hebben tot de code of de documenten, kunnen via de website worden ingediend. De website zal verder ook een overzicht geven (status, beschrijving, …) van alle ingediende bugs of problemen, zij het of deze werden ingediend via de website zelf of via GitHub.
\\
\\
Aan de hand van de ingediende commits en de milestones wordt de vooruitgang van het project in kaart gebracht. Dit kan worden gebruikt voor evaluaties of het bewerken van het requirements dashboard. Ook deze zullen allemaal in de website worden geïntegreerd.
\\
\\
Om grote problemen of code-verlies tegen te gaan, wordt er enerzijds vertrouwd op de berekenbaarheid en stabiliteit van GitHub, en worden er door de Configuration Manager anderzijds wekelijks back-ups genomen van alle documenten en repositories. Deze back-ups zullen ook via de website ter beschikken worden gesteld.

\subsection{Groei en planning}
De in de vorige secties van \ref{SoftwareConfigurationManagementPlan} aangehaalde delen, worden gebruikt in functie van het project en zullen bijgevolg (mogelijk) veranderingen of uitbreidingen ondervinden. Het in dit document beschreven SCMP is dan ook een basis voor goed samenwerken en een richtlijn bij de collaboratie. Om flexibele groei toe te staan in correspondentie met het project, stellen we volgende beslissingsprocedures op voor de wijziging van software, implementatie van nieuwe functies of ingebruikname van nieuwe tools.
\\
\\
Het al dan niet adopteren of afkeuren van functies of software, of de manier waarop deze worden gebruikt, zal altijd in samenspraak met het gehele team gebeuren.
\\
\\
De verdere planning en diepere uitwerking van het gebruik, zal verder vorm gegeven worden in de beginfase van de implementatie.

\subsection{SCM Resources}
Omdat het gebruikte besturingssysteem afhankelijk is van de gebruiker en kan verschillen per lid, wordt er geopteerd voor software die op zijn minst de laatste versies van de besturingssystemen van Windows en Apple ondersteunt.
\\
\\
Er wordt enkel gebruikt gemaakt van open source software of zelf ontwikkelde tools.
\\
\\
Voor de samenwerkingen maken we, zoals eerder vermeld, gebruikt van onze website, waar we trachten alles te centraliseren, en van GitHub, dat voor elk platform beschikbaar is: meer info op \url{http://windows.github.com} en \url{http://mac.github.com} of in sectie \ref{sec:SCMActiviteiten}. De Configuration Managers, of leden die geavanceerder aan de repositories willen sleutelen, zullen dan ook gebruik maken van SourceTree \url{http://www.sourcetreeapp.com}.
\\
\\
Code zal worden geschreven in Eclipse\cite{Eclipse} en gedocumenteerd in Javadoc\cite{Javadoc}.
\\
\\
Al deze software zal worden gebruikt volgens de richtlijnen beschreven in dit document en de opdracht.

\subsection{Onderhoud}
Alle versies van de documenten zullen worden bijgehouden op GitHub en onder toezicht worden geplaatst van o.a. de Configuration Manager.
\\
\\
Na nieuwe beslissingen te hebben genomen in verband met de configuratie, zal dit worden gereflecteerd in het SCMP dat zo snel mogelijk up to date dient te worden gebracht door de Configuration Manager.
\\
\\
Deze versie wordt gereviseerd door de groep en goedgekeurd alvorens het opnieuw in het SPMP wordt opgenomen.

\section{Verificatie en validatie plan}
Hiervoor verwijzen we naar het Software Test Plan (STD) beschikbaar op de website \cite{portalWebsite}.

\section{Documentatie plan} \label{sec:DocumentationPlan} % korte beschrijving
Bij dit project worden de volgende documenten verwacht:
\begin{itemize}
	\item Software Project Management Plan (SPMP)
	    \begin{itemize}
	        \item Software Quality Assurance plan (SQAP als onderdeel van het SPMP)
	        \item Software Configuration Management Plan (SCMP, ook onderdeel van het SPMP)
	    \end{itemize}
	\item Software Test Plan (STD)
	\item Software Requirements Specification (SRS)
	\item Software Design Document (SDD)
	\item Minutes van alle vergaderingen
	\item Documentatie bij de source code
\end{itemize}
De layout van de documenten is vastgelegd volgens de VUB-huisstijl \cite{VUBHuisstijl}, inclusief font-stijl. Het SPMP wort geschreven door de Project Manager, het SQAP en STD door de Quality Assurance Manager, het SRS door de Requirements Manager en het SDD door de Design Manager. Minutes van vergaderingen worden opgesteld door de secretaris. De documentatie van code wordt opgesteld door alle programmeurs en gecontroleerd door de Quality Assurance Manager. Overigens worden zowel source code als alle documenten op kwaliteit gecontroleerd door de Quality Assurance Manager. 
\\
\\
De documenten zullen gestockeerd worden op een afzonderlijke repository op GitHub (behalve de documentatie bij de source code). Er zal gebruik worden gemaakt van de webpagina \cite{portalWebsite} voor het rapporteren en controleren van veranderingen aan de verscheidene documenten. Dit zal gebeuren via de GitHub API \cite{GitHubAPI}, hierdoor worden deze wijzigingen ook doorgevoerd op GitHub.

\section{Software Quality Assurance Plan}
Op vergaderingen met het team wordt besproken of men nog steeds op schema zit van het voorgestelde ontwikkelingsproces. Zo niet moet de Software Quality Assurance Manager proberen te achterhalen wat er mis is gegaan, en hoe het team moet worden bijgestuurd om het gekozen proces wel te volgen.

\subsection{Documenten}
Per iteratie worden 4 (argumenteerbaar 6) documenten opgeleverd die elk door een verschillend persoon worden gemaakt, nl. het SPMP, STD, SRS, SDD, en als onderdeel van het SPMP: het SQAP en SCMP. Al deze documenten worden gemaakt door hun respectievelijke verantwoordelijke. Daarom is het belangrijk dat de Quality Assurance Manager standaarden vastlegt i.v.m. structuur en opmaak. Alle documenten worden geschreven in LaTeX en hun structuur, opmaak, spelling en zinsbouw worden gecontroleerd door de Quality Assurance Manager. Als rechtstreeks gevolg zijn er intern deadlines vastgelegd één week voor officiële deadlines. Dit geeft het hele team ruim voldoende tijd om eventuele gebreken op te lossen, maar ook kan de Quality Assurance Manager documenten en source code controleren voor oplevering. Wanneer een document klaar is dient de verantwoordelijke van dit document een mail te sturen (met rechtstreekse URL) via de mailinglijst naar alle leden van de groep. Vervolgens kan de Quality Assurance Manager controleren of het document voldoet aan de afgesproken layout en conventies. Het is niet de verantwoordelijkheid van de Quality Assurance Manager om documenten te corrigeren, slechts om te controleren en de verantwoordelijke op de hoogte te stellen van eventuele gebreken. Na goedkeuring kunnen documenten door hun respectievelijke verantwoordelijke op GitHub worden geplaatst.

\subsection{Broncode}
\subsubsection{Documentatie en commentaar}
Er wordt verwacht dat de programmeurs hoogkwalitatieve code schrijven, d.w.z. mooie, leesbare code met voldoende commentaar. Bovendien moet code voldoende gedocumenteerd worden d.m.v. JavaDoc voor Java, of equivalent voor andere gebruikte programmeertalen. Deze documentatie moet samen met de commentaar bij broncode voldoende zijn voor de Quality Assurance Manager om code te lezen en begrijpen. Wanneer code niet voldoet aan de vooropgestelde eisen zal de Quality Assurance Manager de programmeur hier onmiddelijk van verwittigen, waarna de programmeur ervoor moet zorgen dat de code wel voldoet aan de eisen. De Quality Assurance Manager zal broncode controleren nadat de programmeur een module heeft afgewerkt. 

Uiteindelijk ligt de verantwoordelijkheid voor hoogkwalitatieve code bij de programmeur zelf. De Quality Assurance Manager zorgt er slechts voor dat code voldoet aan een bepaalde standaard, en zo niet moet de programmeur zijn best doen om deze standaard alsnog te halen. 

\subsubsection{Unittests}
De Quality Assurance Manager zal er ook op toezien dat modules voorzien zijn van geautomatiseerde tests d.m.v. JUnit voor Java of equivalent voor andere programmeertalen. Wanneer deze tests niet aanwezig zijn voor modules die dit wel vereisen zal de programmeur hierop attent gemaakt worden. Code moet dus voldoen aan het Software Test Document (STD).

\subsection{Inspecties}
In onderstaande tabel vindt u de datum van alle uitgevoerde inspecties, de datum waarop veranderingen aan de code zijn toegepast, alsook een korte beschrijving van het onderdeel dat onderhevig is aan inspectie en als laatste een link naar het rapport van de inspectie.

\begin{center}
    \begin{tabular}{| l | l | l | l |}
    \hline
    Inspectie & Doorvoering veranderingen & Beschrijving & Link \\ \hline
    
    \end{tabular}
\end{center}

%\section{Reviews and audits plan} niet

\section{Problem resolution plan} \label{sec:ProblemResolutionPlan}
%wat als er inconsisties met documentatie ? wat met bugfixing ? (gebruik maken van bugtracker tool !!) hoe integratie van oplossingen hiervoor, hoe terug testen ? hier moeten ook procedures voor geschreven worden.
Wanneer er inconsistenties gevonden worden in de documenatie of code zullen deze gemeld worden op GitHub m.b.v. issues. Deze issues worden automatisch ook gesynchroniseerd met de website \cite{portalWebsite}. Deze synchronisatie gebeurt met de GitHub API \cite{GitHubAPI}. Het voordeel van gebruik te maken van de GitHub API is dat alles gecentraliceerd blijf op onze website. Hierdoor is het gemakkelijk om snel een overzicht te krijgen van de huidige stand van zaken. 
\\
\\
Vervolgens wordt er een persoon aangewezen die de verantwoordelijkheid krijgt. Voor de documenten is de verantwoordelijke steeds de verantwoordelijke voor het betreffende document (zie hoofdstuk \ref{chap:ProjectOrganisatie}). Voor issues in de code wordt de documentatie bij de desbetreffende code geraadgepleegd voor de verantwoordelijke aan te wijzen.
\\
\\
De aangewezen verantwoordelijke is dan verantwoordelijk voor het oplossen van het issue. Dit betekent niet noodzakelijk dat deze persoon het issue effectief moet oplossen. Hij mag ook opdrachten doorgeven aan andere teamleaden op het issue op te lossen. De verantwoordelijke moet er enkel voor zorgen dat het issue opgelost geraakt, ongeacht de gebruikte methode. Vervolgens wordt het issue gesloten door deze verantwoordelijke.
%\section{Subcontractor management plans} niet

%\section{Process improvement plan}