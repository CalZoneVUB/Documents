\chapter{Overzicht}
\section{Samenvatting van het project}
\subsection{Doel, scope en objectieven}
Het doel van dit project is het maken van een webapplicatie die het mogelijk maakt om lessenroosters binnen de universiteit te schedulen. Deze lessenroosters moeten vervolgens door de studenten geraadpleegd kunnen worden. Er is verder specifieke support voor mobiele platformen nodig. 
\\
\\
Als projectnaam is gekozen voor ``CalZone'', ge\"{i}spireerd op het feit dat we een zone voor een kalender moeten maken. Het logo is afgebeeld in figuur \ref{fig:logoProject}.
\begin{figure} [H]
    \centering
    \includegraphics[width = 0.75\textwidth]{Overview/logo_Green_Crop.jpg}
    \caption{Het logo.}
    \label{fig:logoProject}
\end{figure}
%\subsection{Onderstellingen en beperkingen}

\subsection{Project deliverables}
De deliverables voor dit project zijn weergegeven in tabel \ref{tab:kalender}.
\begin{table}[H]
  \centering
  \caption{Kalender}
    \begin{tabular}{c|c}
    \textbf{Datum} & \textbf{To Do} \\
    \hline
    Maandag 04/11/2013 & Inleveren SPMP \\
    Vrijdag 15/11/2013 & Eerste versie documenten \\
    Vrijdag 13/12/2013 & Einde iteratie 1: opleveren code en documenten \\
    Woensdag 18/12/2013 & Presentatie \\
    \hline
    \hline
    Dinsdag 04/03/2014 & Einde iteratie 2: opleveren code en documenten \\
    Woensdag 12/03/2014 & Presentatie \\
    Dinsdag 15/04/2014 & Einde iteratie 3: opleveren code en documenten \\
    Woensdag 23/04/2014 & Presentatie \\
    Vrijdag 16/05/2014 & Einde iteratie 4: opleveren code en documenten \\
    Woensdag 21/05/2014 & Finale presentatie \\
    \end{tabular}
  \label{tab:kalender}
\end{table}
Hierbij worden telkens volgende documenten verwacht:
\begin{itemize}
	\item Software Project Management Plan (SPMP)
	\item Software Test Plan (STD)
	\item Software Requirements Specification (SRS)
	\item Software Design Document (SDD)
	\item Minutes van alle vergaderingen
\end{itemize}

\section{Evolutie van het SPMP}
De evoluatie van de SPMP zal bijgehouden worden met behulp van een versie geschiedenis in het begin van dit document. Er zullen steeds geplande updates uitgevoerd worden op de tijdsstippen beschreven in tabel \ref{tab:kalender}. 
\\
\\
Deze updates worden uitgevoerd met behulp van de GitHub Repository \cite{GitHubRepository}. Met behulp van GitHub kunnen we met issues werken, en hiervoor telkens een verantwoordelijke aanduiden.