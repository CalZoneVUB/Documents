\chapter{Introductie}

\section{Doel en scope}
Dit document beschrijft de softwarearchitectuur en het design van de CalZone webapplicatie.
CalZone is een webapplicatie om lessenroosters te plannen en bezichtigen.
Dit document zal gebruikt worden door de programmeurs, de designers en de testers van dit project als naslagwerk en documentatie om de werking van het systeem te vatten. 
Hierdoor kan dit document gebruikt worden om uitbreidingen en aanpassingen aan het systeem mogelijk te maken.


\section{Acroniemen}

\begin{table}[H]
	\centering
	\caption{Acroniemen}
	\label{tab:Acroniemen}
	\begin{tabular}{l | l}
	
	API	& Application Programming Interface\\
	
	DAO	& Data Access Object\\

	DB	& Database\\
	
	EE	& Enterprise Edition\\

	GUI	& Graphical User Interface\\
	
	HTML	& HyperText Markup Language\\

	JSP & Java Server Page\\
	
	IDE	& Integrated Development Environment\\

	MVC & Model View Controller\\ 

	SDD	& Software Design Description\\

	SDK	& Software Development Kit\\

	SRS	& Software Requirements Specification\\
	
	XML & Extensible Markup Language\\
	
	\end{tabular}
\end{table}


\section{Overzicht}
Dit document volgt de IEEE Std 1016-2006\texttrademark \space standaard voor het opstellen van Software Design Descriptions. 
Dit document is be\"{i}nvloed door de requirements beschreven in de Software Requirements Specification van dit project.\cite{SRS}\\ 
In de huidige fase van dit document wordt het design van het systeem in de derde iteratie beschreven. 
In hoofdstuk~\ref{chap:architectuur} wordt de gebruikte systeemarchitectuur toegelicht en in hoofdstuk~\ref{chap:viewpoints} worden verschillende 'design viewpoints' besproken die relevant zijn voor het systeem.