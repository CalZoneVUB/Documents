\section{Logica}
\label{sec:logica}
In deze sectie worden verschillende modules en packages besproken die deel uitmaken van het huidige logische niveau van het systeem.

\subsection{Database Repositories}
\label{subsec:databaseRepo}
CalZone maakt gebruik van een relationele databank, MySQL, als back-end voor dataopslag. 
Om informatie vanuit deze databank te lezen en er naartoe te schrijven zijn er enkele packages voorzien die een abstractie bieden voor het openen en sluiten van de connectie met de databank en het sturen van queries naar de databank.\\ 

Dit is de eerste package die bijdraagt tot deze abstractielaag.
Hier worden allerhande interfaces aangemaakt voor verschillende objecten.

\subsection{Service}
\label{subsec:service}
Dit is de tweede package die abstractie biedt tussen de databank en het logische niveau van het systeem. 
Het is een verbeterde uitbreiding aan de voorgaande DAO klassen die er in het project bestonden. 
De klassen in deze package voorzien dus functies om specifieke data vanuit de databank op te halen.
Deze klassen voeren queries uit en laden de gevonden data in de daarvoor voorziene objecten.

\subsection{Controllers}
\label{subsec:controllers}
Deze klassen zijn verantwoordelijk om de HTTP-requests te verwerken. 
Deze klassen zorgen dus voor een propagatie van (functionaliteits)verzoeken van de front-end naar de back-end. 
Eveneens zorgen deze klassen ook voor een propagatie van data van de back-end naar de front-end. 
Dit laatste uit zich in content op de webpagina's.
Met andere woorden zorgen de controllers dus voor de mogelijke views en voorzien dus de mogelijkheid om de functionaliteiten opgesomd in de use case diagrammen van sectie~\ref{sec:context} uit te voeren. 

\subsection{Validators}
\label{subsec:validators}
Binnenin het systeem dient sommige data gecontroleerd te worden op geldigheid. 
Deze validatorklassen zijn verantwoordelijk om geldigheid van bepaalde informatie te controleren en aan te geven indien deze info niet geldig is. 

\subsection{Model}
\label{subsec:model}
Deze package bevat alle klassen die deel uitmaken van de business logic van het systeem.

\subsection{Scheduler}
\label{subsec:scheduler}
In deze package zitten de klassen die bijdragen tot het maken en formuleren van lessenroosters.
Ook vind je hier de configuraties voor OptaPlanner.
Er is zowel een configuratiebestand voor de uiteindelijke AI agent die lessenroosters zal trachten te maken op basis van verschillende constraints als voor een benchmarker.
OptaPlanner bezit een benchmarking systeem opdat men verschillende configuraties kan testen op verscheidene datasets.
Deze package bezit ook een file met constraint regels.
Deze file bevat code in de programmeertaal Drools\cite{Drools}.

\subsection{Exception}
\label{subsec:exception}
Zelfgemaakte Exceptions worden verzameld in deze package.