\section{Algoritmes}
\label{sec:algorithms}
\subsection{Scheduling}
\label{subsec:scheduling}
Voor het maken van lessenroosters wordt er gebruik gemaakt van de library OptaPlanner\cite{optaplanner}. 
Dit is een constraint solver library met focus op het oplossen van planningen.
Om gebruik te maken van OptaPlanner dien je eerst je planningsprobleem te modelleren.
Dit werd besproken in \ref{subsec:scheduleclass}.\\

De klasse SchedulerSolver representeert de agent die een lessenrooster zal construeren vanuit een lijst van lokalen, datums en vakonderdelen.
Dit doet het door middel van local search algoritmes en construction heuristics die door de programmeur geconfigureerd dienen te worden.
Initieel begint men de agent met een na\"{i}eve opstelling van een Schedule.
De agent berekent dan de score van deze opstelling.
Deze score wordt berekend door na te trekken welke constraints er geschonden werden.
Deze constraints dienen door de programmeur ontwikkeld te worden.
Indien deze score nog niet optimaal is, zal de agent een verschuiving toepassen in het rooster.
Hiermee wordt getracht de score te verbeteren, met andere woorden minder constraints te schenden.
De agent kiest de 'beste' verschuiving en het algoritme herhaalt zich.\\

De exacte manier van het maken van deze verschuivingen en het selecteren van de beste verschuiving in het rooster hangt af van local search algoritmes die gebruikt worden voor het schedulen.
Om de configuratie zo optimaal mogelijk te maken bezit OptaPlanner een benchmarking systeem waarvoor de configuratie terug te vinden is in de package 'Scheduler'.\\

De huidige configuratie van SchedulerSolver gebruik van een construction heuristic, first fit decreasing, gevolgd door een local search algoritme, simulated annealing. 
Als scoresysteem wordt er gebruik gemaakt van een 2-laagse score, beter gekend als Hard-Soft score.
Men kan de constraints dus opdelen in 2 groepen.
Elke constraint bezorgt zijn score en de scores van constraints in dezelfde groep wordt bij mekaar opgeteld.\\

\noindent Volgende hard constraints werden ontwikkeld:
\begin{itemize}
\item Leraren mogen/kunnen geen 2 verschillende lessen op eenzelfde tijdstip geven.
\item In een traject mogen geen lessen overlappen.
\item Een les dient door te gaan in een lokaal met voldoende ruimte.
\item Er kunnen niet meerdere lessen tegelijk doorgaan in eenzelfde lokaal
\item Lessen worden gegeven in de daarvoor voorziene periode/weken.
\item Een les wordt gehouden in een lokaal me de juiste accomodaties voor die les (projectoren, computers...)
\item Lessen moeten wekelijks op hetzelfde moment gepland worden.
\end{itemize}

\noindent Volgende soft constraints werden ontwikkeld:
\begin{itemize}
\item Lessen worden gepland rekening houdend met de voorkeuren die de lesgever heeft opgegeven voor een vak.
\item Lessen worden niet gepland op momenten dat de lesgever niet kan. (Opgegeven door de lesgever).
\item Er worden geen 2 sessies (hoorcolleges of werkcolleges) van hetzelfde vak achter elkaar gepland.
\item Een student heeft maximaal 9 uur les op een dag.
\item Springuren worden vermeden, maar men tracht toch een middagpauze te behouden.
\item Er worden geen lessen gepland die eindigen na 19u.
\end{itemize}

\noindent Deze constraints werden geschreven in de declaratieve taal Drools\cite{Drools}.