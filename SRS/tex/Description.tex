\chapter{Algemene beschrijving}

\section{Gebruikers karakteristieken}
De applicatie zal drie soorten gebruikers ondersteunen:

\begin{enumerate} 
    \item Studenten: Deze zijn “gewone” gebruikers. 
    Ze moeten zich aanmelden om hun vakken te bezichtigen en hebben dan de mogelijkheid om hun vakken pakket aan te passen.
    \item Professoren/assistenten: Deze kunnen hetzelfde als studenten, maar hebben hier bovenop ook nog de mogelijkheid om berichten naar     groepen van studenten te sturen en vakken te verplaatsen van tijdstip of lokaal.
    \item Administrators: Zijn verantwoordelijk voor het onderhouden van de database en de server.
\end{enumerate}

\section{Product perspectief}

\subsection{Software Interfaces}
“CalZone” zal zich verbinden met de Wilma server om daaruit data te weergeven. 
De communicatie zal zowel gewoon data uitlezen toestaan als data modificatie.
De database zal beheerd worden door de database manager.

\subsection{User interfaces}
De user interface zal zowel in het Engels als in het Nederlands beschikbaar zijn. 
Elke gebruiker kiest bij het aanmaken van een account zijn voorkeur taal en deze zal worden opgeslagen, zodat bij het inloggen altijd de juiste taal gebruikt wordt.
\\
\\
Wanneer een user voor de eerste keer “Calzone” gebruikt zal deze zich moeten registreren. 
Als eerst zal hij dan naar een pagina gestuurd worden waar persoonlijke data ingevoerd kan worden (naam, voornaam, e-mail, rolnummer, telefoon, adres, etc.). 
De gebruiker kiest dan zijn richting en jaar en zal daardoor automatisch ingeschreven zijn voor de bijhorende vakken (deze kan de gebruiker dan zelf aanpassen). 
Is de gebruiker al geregistreerd dan kan deze gewoon inloggen met zijn username en paswoord.  

\subsection{Hardware Interfaces}

\section{User Requirements}
Deze sectie hanteert de verschillende acties die de software moet ondersteunen voor verschillende gebruiker.

\subsection{Niet-aangemelde gebruiker}
\begin{enumerate}
    \item Profiel Aanmaken
    \item Aanmelden
    \item Wachtwoord vergeten
    \item Lesrooster van een richting bekijken
    \item Lesrooster van specifiek vak bekijken
    \item Lesrooster van een lokaal bekijken
\end{enumerate}

\subsection{Aangemelde student}
\begin{enumerate}
    \item Ingeschreven vakken bekijken
    \item Ingeschreven vakken aanpassen
    \item Gepersonaliseerd lesrooster bekijken
    \item Taal aanpassen
    \item Afmelden
    \item Profiel gegevens aanpassen
    \item Notificaties van last-minute veranderingen krijgen
\end{enumerate}

\subsection{Aangemelde assistent}
\begin{enumerate}
    \item Gepersonaliseerde Lesrooster bekijken (Les geven)
    \item Taal aanpassen
    \item Afmelden
    \item Profiel gegevens aanpassen
    \item Last-minute tijdstip vak aanpassen 
    \item Beschikbaarheidsformulier aanmaken
    \item Beschikbaarheidsformulier aanpassen
\end{enumerate}

\subsection{Aangemelde professor}
\begin{enumerate}
    \item Taal aanpassen
    \item Last-minute tijdstip les aanpassen 
    \item Locatie les aanpassen
    \item Details les aanpassen
    \item Schema van lesgever bekijken
    \item Detail formulier nieuw vak indienen
    \item Aanpassen detail formulier eigen vak
    \item Beschikbaarheidsformulier aanmaken
    \begin{enumerate}
        \item Aantal betrokken lesgevers aanpassen (assistenten)
        \item Onderverdeling HOC / WPO in aantal uren
        \item Nodige faciliteiten HOC / WPO (Projector, Computers ...)
        \item Lesgever HOC en WPO
        \item Opmerking voor programmabeheerder
    \end{enumerate}
    \item Beschikbaarheidsformulier aanpassen
    \item Afmelden
    \item Aanpassen planning vak
    \item Planning markeren als voorstel
\end{enumerate}

\subsection{Aangemelde programmabeheerder}
\begin{enumerate}
    \item Vakken toevoegen op basis van ingediend detail formulier
    \item Vakken verwijderen van lesprogramma
    \item Aanpassen detail formulier specifiek vak
    \item Lokaal detail formulier toevoegen
    \begin{enumerate}
        \item Beschikbare faciliteiten (Projector, Computers ...)
        \item Vrije uren
        \item Aantal zitplaatsen
        \item Opmerkingen
    \end{enumerate}
    \item Lokaal detail formulier verwijderen
    \item Lokaal detail formulier aanpassen
    \item Lesprogramma aanmaken voor een richting
    \item Bekijken lesserooster als specifiek gebruiker
    \item Bevriezen van een voorstel
    \item Bevriezen volledige planning
\end{enumerate}

\subsection{Scheduler}
\begin{enumerate}
    \item Uitvoeren van een full scheduling
    \item Professor, assistent en programmabeheerder moeten semi-scheduling laten uitvoeren (Haalbaarheid) 
    \item Schedule bevat data, tijdstippen en lokalen voor bepaalde vakken
    \item Aanpassen berekend rooster
    \item Conflictdetectie
\end{enumerate}