\subsection{Aangemelde student - FR2}

\noindent\begin{table}[h]
            \begin{tabular}{l | p{10cm}}
                \textbf{ID:} & FR2.1 \\ \hline
                \textbf{TITEL:} & Ingeschreven vakken bekijken\\ \hline
                \textbf{PRIORITEIT:} &  High \\ \hline
                \textbf{PREREQUISITIES:} & \\ \hline
                \textbf{BESCHRIJVING:} & Een aangemelde student kan ten alle tijden de lijst van vakken zien waarvoor deze is ingeschreven. 
                                        De student drukt hiervoor op een knop getiteld “Ingeschreven Vakken” en wordt dan naar een pagina doorgestuurd waarop deze in lijstvorm te zien zijn.\\
            \end{tabular}\\
            \caption{FR2.1}
            \label{tab:myseventhtable}
        \end{table}

%\rule{\textwidth}{.4pt}\\

\noindent\begin{table}[h]
            \begin{tabular}{l | p{10cm}}
                \textbf{ID:} & FR2.2 \\ \hline
                \textbf{TITEL:} & Ingeschreven vakken aanpassen\\ \hline
                \textbf{PRIORITEIT:} &  High \\ \hline
                \textbf{PREREQUISITIES:} & \\ \hline
                \textbf{BESCHRIJVING:} & Een aangemelde student kan de vakken waarvoor deze is ingeschreven aanpassen (zolang de VUB dit toelaat!). 
                                        De student kan zich uitschrijven voor een vak door te klikken op de knop getiteld “Uitschrijven” die zich naast het vak bevind waarvoor deze zich wil uitschrijven. 
                                        Er zal dan een window geopend worden die nog vraagt of de gebruiker zeker is, die uitgerust is met een “ja” en “nee” knop. 
                                        Om zich in te schrijven voor een vak moet de student een vak opzoeken. 
                                        Hiervoor moet eerst de richting opgegeven waar het vak toe behoort en dan kan de gebruiker in de lijst van vakken zoeken op naam en vaknummer en wordt dan naar een pagina doorgestuurd waarop deze in lijstvorm te zien zijn.\\
            \end{tabular}\\
            \caption{FR2.2}
            \label{tab:myeighthtable}
        \end{table}

%\rule{\textwidth}{.4pt}\\

\noindent\begin{table}[h]
            \begin{tabular}{l | p{10cm}}
                \textbf{ID:} & FR2.3 \\ \hline
                \textbf{TITEL:} & Gepersonaliseerd lesrooster bekijken\\ \hline
                \textbf{PRIORITEIT:} &  High \\ \hline
                \textbf{PREREQUISITIES:} & \\ \hline
                \textbf{BESCHRIJVING:} & Een aangemelde student kan zijn eigen lesrooster bekijken, deze bevat alleen de vakken waarvoor de student ingeschreven is. 
                                        Het lesrooster geeft aan waar en om hoe laat een vak gegeven wordt en kan waarschuwingen geven bij mogelijke overlappingen.\\ 
            \end{tabular}\\
            \caption{FR2.3}
            \label{tab:myninthtable}
        \end{table}

%\rule{\textwidth}{.4pt}\\

\noindent\begin{table}[h]
            \begin{tabular}{l | p{10cm}}
                \textbf{ID:} & FR2.4 \\ \hline
                \textbf{TITEL:} & Taal aanpassen\\ \hline
                \textbf{PRIORITEIT:} &  Laag \\ \hline
                \textbf{PREREQUISITIES:} & \\ \hline
                \textbf{BESCHRIJVING:} & Een aangemelde student kan zijn taal van voorkeur aanpassen. 
                                        Hiervoor drukt de gebruiker op de knop getiteld “Taal”, hierna zal er een keuze zijn aan talen waarin “CalZone” beschikbaar is.\\ 
            \end{tabular}\\
            \caption{FR2.4}
            \label{tab:mytenthtable}
        \end{table}

%\rule{\textwidth}{.4pt}\\

\noindent\begin{table}[h]
            \begin{tabular}{l | p{10cm}}
                \textbf{ID:} & FR2.5 \\ \hline
                \textbf{TITEL:} & Afmelden\\ \hline
                \textbf{PRIORITEIT:} &  Hoog \\ \hline
                \textbf{PREREQUISITIES:} & \\ \hline
                \textbf{BESCHRIJVING:} & Om af te melden moet een gebruiker simpelweg op de knop getiteld “Afmelden” drukken die de gebruiker terug naar de inlog pagina van het programma brengt.\\
            \end{tabular}\\
            \caption{FR2.5}
            \label{tab:myeleventhtable}
        \end{table}

%\rule{\textwidth}{.4pt}\\

\noindent\begin{table}[h]
            \begin{tabular}{l | p{10cm}}
                \textbf{ID:} & FR2.6 \\ \hline
                \textbf{TITEL:} & Profiel gegevens aanpassen\\ \hline
                \textbf{PRIORITEIT:} &  medium \\ \hline
                \textbf{PREREQUISITIES:} & \\ \hline
                \textbf{BESCHRIJVING:} & Een gebruiker kan zijn profielgegevens aanpassen door te drukken op de knop getiteld “Profiel”. 
                                        En kan dan de volgende informatie aanpassen:
                                        \begin{itemize}\itemsep1pt \parskip0pt \parsep0pt
                                        \item Naam
                                        \item Voornaam
                                        \item Gebruikersnaam
                                        \item Wachtwoord
                                        \item Geboortedatum
                                        \item E-mail
                                        \item Rolnummer
                                        \item Richting (= vaste lijst)
                                        \end{itemize}\\
            \end{tabular}\\
            \caption{FR3.6}
            \label{tab:mytwelfthtable}
        \end{table}



%\rule{\textwidth}{.4pt}\\

\noindent\begin{table}[h]
            \begin{tabular}{l | p{10cm}}
                \textbf{ID:} & FR2.7 \\ \hline
                \textbf{TITEL:} & Notificaties van last-minute veranderingen krijgen\\ \hline
                \textbf{PRIORITEIT:} &  medium \\ \hline
                \textbf{PREREQUISITIES:} & \\ \hline
                \textbf{BESCHRIJVING:} & Mocht er een lesaanpassing plaats vinden (bijv.: een professor die de les van lokaal verandert) dan moeten studenten die dat vak volgen verwittigd worden. 
                                        Zij zullen dan een bericht op hun e-mail adres ontvangen met daarin de nodige informatie over de lesaanpassing.\\
            \end{tabular}\\
            \caption{FR2.7}
\label{tab:mythirteenthtable}
        \end{table}
\clearpage