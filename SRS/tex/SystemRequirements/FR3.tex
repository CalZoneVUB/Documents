\subsection{Aangemelde assistent - FR3}

\noindent\begin{table}[h]
            \begin{tabular}{l | p{10cm}}
                \textbf{ID:} & FR3.1 \\ \hline
                \textbf{TITEL:} & Gepersonaliseerd lesrooster bekijken\\ \hline
                \textbf{PRIORITEIT:} &  High \\ \hline
                \textbf{PREREQUISITIES:} & \\ \hline
                \textbf{BESCHRIJVING:} & Een aangemelde assistent kan zijn eigen lesrooster bekijken, deze bevat alleen de vakken waar deze les moet geven. 
                                        Het lesrooster geeft aan waar en om hoe laat een vak gegeven wordt en kan waarschuwingen geven bij mogelijke overlappingen.\\ 
            \end{tabular}\\
            \caption{FR3.1}
            \label{tab:myfourteenthtable}
        \end{table}
        
\noindent\begin{table}[h]
            \begin{tabular}{l | p{10cm}}
                \textbf{ID:} & FR3.2 \\ \hline
                \textbf{TITEL:} & Taal aanpassen\\ \hline
                \textbf{PRIORITEIT:} &  Laag \\ \hline
                \textbf{PREREQUISITIES:} & \\ \hline
                \textbf{BESCHRIJVING:} & Een aangemelde assistent kan zijn taal van voorkeur aanpassen. 
                                        Hiervoor drukt de gebruiker op de knop getiteld “Taal”, hierna zal er een keuze zijn aan talen waarin “CalZone” beschikbaar is.\\ 
            \end{tabular}\\
            \caption{FR3.2}
            \label{tab:myfifteenthtable}
        \end{table}
        
\noindent\begin{table}[h]
            \begin{tabular}{l | p{10cm}}
                \textbf{ID:} & FR3.3 \\ \hline
                \textbf{TITEL:} & Afmelden\\ \hline
                \textbf{PRIORITEIT:} &  Hoog \\ \hline
                \textbf{PREREQUISITIES:} & \\ \hline
                \textbf{BESCHRIJVING:} & Om af te melden moet een gebruiker simpelweg op de knop getiteld “Afmelden” drukken die de gebruiker terug naar de inlog pagina van het programma brengt.\\
            \end{tabular}\\
            \caption{FR3.3}
            \label{tab:mysixteenthtable}
        \end{table}
        
\noindent\begin{table}[h]
            \begin{tabular}{l | p{10cm}}
                \textbf{ID:} & FR3.4 \\ \hline
                \textbf{TITEL:} & Profiel gegevens aanpassen\\ \hline
                \textbf{PRIORITEIT:} &  medium \\ \hline
                \textbf{PREREQUISITIES:} & \\ \hline
                \textbf{BESCHRIJVING:} & Een gebruiker kan zijn profielgegevens aanpassen door te drukken op de knop getiteld “Profiel”. 
                                        En kan dan de volgende informatie aanpassen:
                                        \begin{itemize}\itemsep1pt \parskip0pt \parsep0pt
                                            \item Naam
                                            \item Voornaam
                                            \item Gebruikersnaam
                                            \item Wachtwoord
                                            \item Geboortedatum
                                            \item E-mail
                                            \item Rolnummer
                                            \item Richting (= vaste lijst)
                                        \end{itemize} \\
            \end{tabular}\\
            \caption{FR3.4}
            \label{tab:myseventeenthtable}
        \end{table}

\noindent\begin{table}[h]
            \begin{tabular}{l | p{10cm}}
                \textbf{ID:} & FR3.5 \\ \hline
                \textbf{TITEL:} & Last-minute tijdstip vak aanpasen\\ \hline
                \textbf{PRIORITEIT:} &  medium \\ \hline
                \textbf{PREREQUISITIES:} & \\ \hline
                \textbf{BESCHRIJVING:} & Een assistent heeft de optie om het tijdstip van een vak last-minute aan te passen. 
                                        De assistent drukt dan op de knop getiteld “Lesuur aanpassen” en wordt dan gevraagd om een nieuwe dag en een tijdstip uit te kiezen. 
                                        Hierna wordt dan een e-mail gestuurd naar alle studenten die dat vak volgen met het nieuwe tijdstip van het vak.\\ 
            \end{tabular}\\
            \caption{FR3.5}
            \label{tab:myeighteenthtable}
        \end{table}
        
\noindent\begin{table}[h]
            \begin{tabular}{l | p{10cm}}
                \textbf{ID:} & FR3.6 \\ \hline
                \textbf{TITEL:} & Beschikbaarheids formulier aanmaken\\ \hline
                \textbf{PRIORITEIT:} &  laag \\ \hline
                \textbf{PREREQUISITIES:} & \\ \hline
                \textbf{BESCHRIJVING:} & Een assistent heeft de optie om een formulier in te dienen waarin hij aangeeft wat zijn vrije momenten zijn om les te geven, de Scheduler houd hier dan rekening mee bij het opstellen van de lesroosters.\\ 
            \end{tabular}\\
            \caption{FR3.6}
            \label{tab:mynineteenthtable}
        \end{table}
        
\noindent\begin{table}[h]
            \begin{tabular}{l | p{10cm}}
                \textbf{ID:} & FR3.7 \\ \hline
                \textbf{TITEL:} & Beschikbaarheids formulier aanpassen\\ \hline
                \textbf{PRIORITEIT:} &  laag \\ \hline
                \textbf{PREREQUISITIES:} & Assistent moet al een beschikbaarheids formulier ingedient hebben\\ \hline
                \textbf{BESCHRIJVING:} & Een assistent heeft de optie om zijn beschikbaarheids formulier aan te passen.\\ 
            \end{tabular}\\
            \caption{FR3.7}
            \label{tab:mytwentiethtable}
        \end{table}
\clearpage