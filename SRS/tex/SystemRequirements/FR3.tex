\subsection{Agenda Functionaliteit - FR3}

%Richting Bekijken    
\noindent\begin{table}[H]
            \begin{tabular}{l | p{10cm}}
                \textbf{ID:} & FR1.4 \\ \hline
                \textbf{TITEL:} & Lesrooster van richting bekijken \\ \hline
                \textbf{PRIORITEIT:} &  High \\ \hline
                \textbf{PREREQUISITIES:} & \\ \hline
                \textbf{TOEGANG:} &  Iedereen \\ \hline
                \textbf{BESCHRIJVING:} & Een (niet) ingelogde gebruiker heeft de optie om het standaard lesrooster van een bepaalde richting te bekijken. Hiervoor moet hij een richting en jaar uitkiezen uit een lijst van alle richtingen.\\
            \end{tabular}\\
            \caption{FR1.4 - Lesrooster van richting bekijken}
            \label{tab:FR1.4 - Lesrooster van richting bekijken}
        \end{table}

%TODO staat er een knop voor "anonieme" sessie?
\textbf{Stappenplan:}
\begin{itemize}
\item Voor de niet-aangemelde gebruiker: 
	\begin{enumerate}
		\item De niet-aangemelde gebruiker (voortaan gebruiker geheten) ziet het beginvenster van de applicatie in zijn webbrowser. Rechts in het scherm ziet hij een optie met de naam "Lessenroosters".
		\item De gebruiker klikt op deze optie en wordt doorverwezen naar een nieuwe pagina. Op deze nieuwe pagina ziet de gebruiker 2 opties om lesroosters te bezichtigen: per richting of per vak.
		\item De gebruiker kiest de optie om per richting roosters te bezichtigen en wordt doorverwezen naar een nieuw venster om waar hij 3 lijsten ziet. Deze lijsten bevatten respectievelijk de faculteiten, richtingen per faculteit en de jaren per richting.
		\item De gebruiker selecteert telkens de gewenste optie voor de 3 lijsten. Het is noodzakelijk dat de gebruiker dit in de juiste volgorde doet: eerst faculteit, vervolgens de richting en tenslotte het jaar. De volgende lijst in deze reeks is namelijk telkens gebaseerd op de vorige selectie(s).
		\item Nadat de gebruiker de gewenste opties heeft geselecteerd, klikt hij op de knop "Toon rooster" om het rooster op te vragen.
		\item Het gevraagde lesrooster verschijnt in hetzelfde venster naast de 3 lijsten.
		\item De gebruiker kan telkens stap 4, 5 en 6 herhalen om andere roosters te tonen. Hierdoor verdwijnt de vorige opzoeking uit het scherm en komt de nieuwe in de plaats.
	\end{enumerate}
\item Voor de aangemelde gebruiker:
	\begin{enumerate}
	\item De aangemelde gebruiker (voortaan gebruiker geheten) ziet het venster met zijn persoonlijk lesrooster in zijn webbrowser. Rechts in het scherm ziet hij een optie met de naam "Lessenroosters".
	\item De gebruiker doorloopt dezelfde stappen van het stappenplan van de niet-aangemelde gebruiker vanaf stap 2.
	\end{enumerate}
\end{itemize}

%TODO
\textbf{Uitzonderingen:}
\begin{itemize}
\item Lesrooster onbeschikbaar
\end{itemize}

%TODO Meerdere zoekopties voorleggen (naam, titularis...)

%Specifiek vak bekijken      
\noindent\begin{table}[H]
            \begin{tabular}{l | p{10cm}}
                \textbf{ID:} & FR1.5 \\ \hline
                \textbf{TITEL:} & Lesrooster van specifiek vak bekijken \\ \hline
                \textbf{PRIORITEIT:} &  High \\ \hline
                \textbf{PREREQUISITIES:} & \\ \hline
                \textbf{TOEGANG:} &  Iedereen \\ \hline
                \textbf{BESCHRIJVING:} & Een (niet)ingelogde gebruiker heeft de optie om het lesrooster van een bepaald vak te bekijken. 
                                        Hiervoor moet hij \'{e}\'{e}n van de bestaande vakken aan de VUB kiezen.\\
            \end{tabular}\\
            \caption{FR1.5 - Lesrooster van specifiek vak bekijken}
            \label{tab:FR1.5 - Lesrooster van specifiek vak bekijken}
        \end{table}

\textbf{Stappenplan:}
\begin{itemize}
\item Voor de niet-aangemelde gebruiker: 
	\begin{enumerate}
		\item De niet-aangemelde gebruiker (voortaan gebruiker geheten) ziet het beginvenster van de applicatie in zijn webbrowser. Rechts in het scherm ziet hij een optie met de naam "Lessenroosters".
		\item De gebruiker klikt op deze optie en wordt doorverwezen naar een nieuwe pagina. Op deze nieuwe pagina ziet de gebruiker 2 opties om lesroosters te bezichtigen: per richting of per vak.
		\item De gebruiker kiest de optie om per vak roosters te bezichtigen en wordt doorverwezen naar een nieuw venster om waar hij een veld ziet om een zoekterm in te vullen.
		\item De gebruiker vult een zoekterm in. Deze zoekterm verwijst naar het vak waarvan de gebruiker het rooster wil bezichtigen.
		\item De gebruiker klikt op de knop met de naam "Zoek vak". Er verschijnt in hetzelfde venster een tabel met gevonden vakken die voldoen aan de zoekterm. In die tabel kan je de naam en de titularis van het vak bezichtigen alsook de richtingen waaraan het vak gedoceerd wordt.
		\item De gebruiker selecteert in de tabel het gewenste vak. Hierdoor verdwijnt de tabel en verschijnt er in de plaats het lesrooster van het geselecteerde vak.
		\item De gebruiker kan telkens stap 4 en 5 herhalen om andere roosters te tonen. Hierdoor verdwijnt de vorige opzoeking uit het scherm en komt de nieuwe in de plaats.
	\end{enumerate}
\item Voor de aangemelde gebruiker:
	\begin{enumerate}
	\item De aangemelde gebruiker (voortaan gebruiker geheten) ziet het venster met zijn persoonlijk lesrooster in zijn webbrowser. Rechts in het scherm ziet hij een optie met de naam "Lessenroosters".
	\item De gebruiker doorloopt dezelfde stappen van het stappenplan van de niet-aangemelde gebruiker vanaf stap 2.
	\end{enumerate}
\end{itemize}

%TODO
\textbf{Uitzonderingen:}
\begin{enumerate}
\item Vak wordt niet gevonden
\end{enumerate}
        
%Lessenrooster Locaal bekijken
\noindent\begin{table}[H]
            \begin{tabular}{l | p{10cm}}
                \textbf{ID:} & FR1.6 \\ \hline
                \textbf{TITEL:} & Lesrooster van een lokaal bekijken\\ \hline
                \textbf{PRIORITEIT:} &  High \\ \hline
                \textbf{PREREQUISITIES:} & \\ \hline
                \textbf{TOEGANG:} &  Iedereen \\ \hline
                \textbf{BESCHRIJVING:} & Een (niet) ingelogde gebruiker heeft de optie om het lesrooster van een leslokaal te bekijken, d.w.z. wanneer dat lokaal bezet is. 
                                        Hiervoor moet de gebruiker zelf een les lokaal invullen van de vorm G.V.L (met G = gebouw, V = verdieping en L = lokaal)\\
            \end{tabular}\\
            \caption{FR1.6 - Lesrooster van een lokaal bekijken}
            \label{tab:FR1.6 - Lesrooster van een lokaal bekijken}
        \end{table}
   
%Ingeschreven vakken bekijken        
\noindent\begin{table}[H]
            \begin{tabular}{l | p{10cm}}
                \textbf{ID:} & FR1.7 \\ \hline
                \textbf{TITEL:} & Ingeschreven vakken bekijken\\ \hline
                \textbf{PRIORITEIT:} &  High \\ \hline
                \textbf{PREREQUISITIES:} & \\ \hline
                \textbf{TOEGANG:} &  Student \\ \hline
                \textbf{BESCHRIJVING:} & Een aangemelde student kan ten alle tijden de lijst van vakken zien waarvoor deze is ingeschreven. De student drukt hiervoor op een knop getiteld “Ingeschreven Vakken” en wordt dan naar een pagina doorgestuurd waarop deze in lijstvorm te zien zijn.\\
            \end{tabular}\\
            \caption{FR1.7 - Ingeschreven vakken bekijken}
            \label{tab:FR1.7 - Ingeschreven vakken bekijken}
        \end{table}  

%Gepersonaliseer lesrooster bekijken
\noindent\begin{table}[H]
            \begin{tabular}{l | p{10cm}}
                \textbf{ID:} & FR1.9 \\ \hline
                \textbf{TITEL:} & Gepersonaliseerd lesrooster bekijken\\ \hline
                \textbf{PRIORITEIT:} &  High \\ \hline
                \textbf{PREREQUISITIES:} & \\ \hline
                \textbf{TOEGANG:} &  Student, Assistent, Professor \\ \hline
                \textbf{BESCHRIJVING:} & Een aangemelde gebruiker kan op elk moment zijn persoonlijk rooster opvragen. Voor studenten gaat dit over lessen volgen. Assistenten en professoren krijgen hier de lessen gepresenteerd die ze moeten geven. Een aangemelde student kan zijn eigen lesrooster bekijken, deze bevat alleen de vakken waarvoor de student ingeschreven is. 
                                        Het lesrooster geeft aan waar en om hoe laat een vak gegeven wordt en kan waarschuwingen geven bij mogelijke overlappingen.\\ 
            \end{tabular}\\
            \caption{FR1.9  - Gepersonaliseerd lesrooster bekijken}
            \label{tab:FR1.9 - Gepersonaliseerd lesrooster bekijken}
        \end{table}

%Notifiacties van last-minute verandering krijgen        
\noindent\begin{table}[H]
            \begin{tabular}{l | p{10cm}}
                \textbf{ID:} & FR1.13 \\ \hline
                \textbf{TITEL:} & Notificaties van last-minute veranderingen \\ \hline
                \textbf{PRIORITEIT:} &  medium \\ \hline
                \textbf{PREREQUISITIES:} & \\ \hline
                \textbf{TOEGANG:} & Student, Assistent, Professor \\ \hline
                \textbf{BESCHRIJVING:} & Mocht er een lesaanpassing plaats vinden (bijv.: een professor die de les van lokaal verandert) dan moeten de gebruiker die dat vak volgen of geven verwittigd worden. Zij zullen dan een bericht op hun e-mail adres ontvangen met daarin de nodige informatie over de lesaanpassing.\\
            \end{tabular}\\
            \caption{FR1.13 - Notificaties van last-minute veranderingen}
			\label{tab:FR1.13 - Notificaties van last-minute veranderingen}
        \end{table}      

%Last-minute tijdstip vak aanpassen
\noindent\begin{table}[H]
            \begin{tabular}{l | p{10cm}}
                \textbf{ID:} & FR1.14 \\ \hline
                \textbf{TITEL:} & Last-minute tijdstip vak aanpassen\\ \hline
                \textbf{PRIORITEIT:} &  medium \\ \hline
                \textbf{PREREQUISITIES:} & \\ \hline
                \textbf{TOEGANG:} & Assistent, Professor, Programmabeheerder \\ \hline
                \textbf{BESCHRIJVING:} & Een gebruiker van type Toegang heeft de optie om het tijdstip van een vak last-minute aan te passen. Enkel toegelaten als deze gebruiker de les zelf geeft of de programmabeheerder is.  De gebruiker drukt dan op de knop getiteld “Lesuur aanpassen” en wordt dan gevraagd om een nieuwe dag en een tijdstip uit te kiezen. 
                                        Hierna wordt dan een e-mail gestuurd naar alle studenten die dat vak volgen met het nieuwe tijdstip van het vak.\\ 
            \end{tabular}\\
            \caption{FR1.14 - Last-minute tijdstip vak aanpassen}
            \label{tab:FR1.14 - Last-minute tijdstip vak aanpassen}
        \end{table}
        
%Locatie Les aanpassen
\noindent\begin{table}[H]
            \begin{tabular}{l | p{10cm}}
                \textbf{ID:} & FR1.15 \\ \hline
                \textbf{TITEL:} & Locatie les aanpasen\\ \hline
                \textbf{PRIORITEIT:} &  medium \\ \hline
                \textbf{PREREQUISITIES:} & \\ \hline
                \textbf{TOEGANG:} & Assistent, Professor \\ \hline
                \textbf{BESCHRIJVING:} & Een professor heeft de optie om de locatie van een vak aan te passen. 
                                        De professor drukt dan op de knop getiteld “Lokaal aanpassen” en wordt dan gevraagd om een nieuwe locatie uit te kiezen, deze locatie is dan van de vorm G.V.L (met G = gebouw, V = verdieping en L = lokaal). De Scheduler zal dan nagaan of het ingevoerde lokaal vrij is op het tijdstip van het vak. Keurt de Scheduler het nieuwe lokaal goed, dan wordt er een e-mail gestuurd naar alle studenten die het vak volgen met daarin het nieuwe lokaal van het vak. Keurt de Scheduler het nieuwe lokaal af, dan wordt de professor gevraagd een ander lokaal in te geven.\\ 
            \end{tabular}\\
            \caption{FR1.15 - Locatie les aanpassen}
            \label{tab:FR1.15 - Locatie les aanpassen}
        \end{table}
%Details les aanpassen 
\noindent\begin{table}[H]
            \begin{tabular}{l | p{10cm}}
                \textbf{ID:} & FR1.17 \\ \hline
                \textbf{TITEL:} & Details vak aanpassen\\ \hline
                \textbf{PRIORITEIT:} &  laag \\ \hline
                \textbf{PREREQUISITIES:} & \\ \hline
                \textbf{TOEGANG:} & Professor, Programmabeheerder \\ \hline
                \textbf{BESCHRIJVING:} & Een professor kan de beschrijving en details van een vak aanpassen. Hiervoor zal er een knop getiteld “Vak details aanpassen” zijn waarna de professor vrij is in het aanpassen van de details. \\ 
            \end{tabular}\\
            \caption{FR1.17 - Details vak aanpassen}
            \label{tab:FR1.17 - Details vak aanpassen}
        \end{table}
 
%Gespecializeerd schema opvragen lesgever.       
\noindent\begin{table}[H]
            \begin{tabular}{l | p{10cm}}
                \textbf{ID:} & FR1.18 \\ \hline
                \textbf{TITEL:} & Gespecialiseerd schema opvragen lesgever\\ \hline
                \textbf{PRIORITEIT:} &  Medium \\ \hline
                \textbf{PREREQUISITIES:} & \\ \hline
                \textbf{TOEGANG:} & Professor \\ \hline
                \textbf{BESCHRIJVING:} & Een professor kan zijn eigen schema van alle lessen bekijken die onder zijn bevoegdheid vallen. Ook WPO's die niet door de professor gegeven worden staan op deze speciale lijst. Verschil word aangeduid via kleurmarkeringen. \\ 
            \end{tabular}\\
            \caption{FR1.18 - Gespecialiseerd schema opvragen lesgever}
            \label{tab:FR1.18 - Gespecializeerd schema opvragen lesgever}
        \end{table}

%Detail formulier nieuw vak indienen
\noindent\begin{table}[H]
            \begin{tabular}{l | p{10cm}} 
                \textbf{ID:} & FR1.19 \\ \hline
                \textbf{TITEL:} & Detail formulier nieuw vak indienen\\ \hline
                \textbf{PRIORITEIT:} &  laag \\ \hline
                \textbf{PREREQUISITIES:} & \\ \hline
                \textbf{TOEGANG:} & Professor \\ \hline
                \textbf{BESCHRIJVING:} & Mocht er nood zijn aan het toevoegen van een nieuw vak aan de database, dan kan een professor hiervoor een formulier opstellen met daarin:
        \begin{itemize}\itemsep1pt \parskip0pt \parsep0pt
                                        \item beschrijving van het vak
                                        \item aantal studiepunten
                                        \item nodige prerequisities
                                        \item welk semester het vak gegeven wordt
                                        \item aantal uren studie tijd
                                        \item studiegidsnummer
                                        \item 2e zittijd mogelijk
                                        \item onderwijstaal
                                        \item faculteit
                                        \item verantwoordelijke vakgroep
                                        \item onderwijsteam
                                        \item onderdelen en contacturen
                                        \item studiemateriaal
                                        \item leerresultaten
                                        \item beoordelingsinformatie
                                        \item Aanvullende info m.b.t. examinering
                                        \item Academische context
                                        \end{itemize}
                                        Het formulier wordt dan gestuurd naar de Programmabeheerder die het vak zal toevoegen. 
            \end{tabular}\\
            \caption{FR1.19 - Detail formulier nieuw vak indienen}
            \label{tab:FR1.19 - Detail formulier nieuw vak indienen}
        \end{table}

%Goedkeuren detail formulier     
\noindent\begin{table}[H]
            \begin{tabular}{l | p{10cm}}
                \textbf{ID:} & FR1.20 \\ \hline
                \textbf{TITEL:} & Goedkeuren detail formulier   \\ \hline
                \textbf{PRIORITEIT:} &  laag \\ \hline
                \textbf{PREREQUISITIES:} & De professor moet al een detail formulier ingedient hebben\\ \hline
                \textbf{TOEGANG:} & Programmabeheerder \\ \hline
                \textbf{BESCHRIJVING:} & De programma beheerder bekijkt en ingediende formulier van een professor en beslist dit toe wel of niet toe te voegen aan het de database. 
            \end{tabular}\\
            \caption{FR1.20 - Goedkeuren detail formulier  }
            \label{tab:FR1.20 - Goedkeuren detail formulier  }
        \end{table}

%Aanpassen detail formulier vak      
\noindent\begin{table}[H]
            \begin{tabular}{l | p{10cm}}
                \textbf{ID:} & FR1.21 \\ \hline
                \textbf{TITEL:} & Aanpassen detail formulier vak \\ \hline
                \textbf{PRIORITEIT:} &  laag \\ \hline
                \textbf{PREREQUISITIES:} & Detailformulier vak moet al bestaan\\ \hline
                \textbf{TOEGANG:} & Professor, Programmabeheerder \\ \hline
                \textbf{BESCHRIJVING:} & Zolang het nieuwe vak nog niet toegevoegd is door de Programmabeheerder kan de professor nog aanpassingen aanbrengen op het ingediende formulier. Ook is het mogelijk an bestaande detailformulieren aanpassingen te doen. Dit mag enkel gebeuren voordat het schema is gemaakt.\\ 
            \end{tabular}\\
            \caption{FR1.21 - Aanpassen detail formulier vak}
            \label{tab:FR1.21 - Aanpassen detail formulier vak}
        \end{table}

%Aanpassen planning vak        
\noindent\begin{table}[H]
            \begin{tabular}{l | p{10cm}}
                \textbf{ID:} & FR1.22 \\ \hline
                \textbf{TITEL:} & Aanpassen planning vak\\ \hline
                \textbf{PRIORITEIT:} &  Medium \\ \hline
                \textbf{PREREQUISITIES:} & \\ \hline
                \textbf{TOEGANG:} & Professor \\ \hline
                \textbf{BESCHRIJVING:} & Het moet mogelijk  zijn een vak te verplaatsen in de algemene planning voor elke week. Dit kan gebeuren op een schema dat al vast ligt en eventueel al bezig is. Als een verplaatsing geen conflicten veroorzaakt gebeurd dit automatisch. Anders wordt een voorstel doorgeschoven aan de programmabeheerder om het goed te keuren.\\
            \end{tabular}\\
            \caption{FR1.22 - Aanpassen planning vak}
            \label{tab:FR1.22 - Aanpassen planning vak}
        \end{table}
        
\noindent\begin{table}[H]
            \begin{tabular}{l | p{10cm}}
                \textbf{ID:} & FR1.23 \\ \hline
                \textbf{TITEL:} & Planning markeren als voorstel\\ \hline
                \textbf{PRIORITEIT:} &  medium \\ \hline
                \textbf{PREREQUISITIES:} & \\ \hline
                \textbf{TOEGANG:} & Professor \\ \hline
                \textbf{BESCHRIJVING:} & Een professor kan een voorstel indienen aan de Scheduler. 
                                        Deze geeft dan in wanneer hij zijn vak het liefst zou geven en dan wordt dit doorgegeven aan de Scheduler die dan rekening probeert te houden met de ingegeven constraint. Dit staat los van de beschikbaarheid van de professor en lokalen.  \\
            \end{tabular}\\
            \caption{FR1.23 - Planning markeren als voorstel}
            \label{tab:FR1.23 - Planning markeren als voorstel}
        \end{table}

%Vakken verwijderen van lesprogramma        
\noindent\begin{table}[H]
            \begin{tabular}{l | p{10cm}}
                \textbf{ID:} & FR1.24 \\ \hline
                \textbf{TITEL:} & Vakken verwijderen van lesprogramma\\ \hline
                \textbf{PRIORITEIT:} &  hoog \\ \hline
                \textbf{PREREQUISITIES:} & \\ \hline
                \textbf{TOEGANG:} & Programmabeheerder \\ \hline
                \textbf{BESCHRIJVING:} & De Programmabeheerder kan ook vakken verwijderen uit lesprogramma’s \\ 
            \end{tabular}\\
            \caption{FR1.24 - Vakken verwijderen van lesprogramma}
            \label{tab:FR1.24 - Vakken verwijderen van lesprogramma}
        \end{table}

   
%Lokaal detailformulier toevoegen  
\noindent\begin{table}[H]
            \begin{tabular}{l | p{10cm}}
                \textbf{ID:} & FR1.25 \\ \hline
                \textbf{TITEL:} & Lokaal detailformulier toevoegen\\ \hline
                \textbf{PRIORITEIT:} &  hoog \\ \hline
                \textbf{PREREQUISITIES:} & \\ \hline
                \textbf{TOEGANG:} & Programmabeheerder \\ \hline
                \textbf{BESCHRIJVING:} & De Programmabeheerder moet in staat zijn om lokalen toe te voegen aan de databse zodat de Scheduler die kan gebruiken in zijn planning. Lokalen hebben een ID nummer van de vorm G.V.L (met G = gebouw, V = verdieping en L = lokaal) en hebben ook een maximaal aantal plaats.\\ 
            \end{tabular}\\
            \caption{FR1.25 -Lokaal detailformulier toevoegen}
            \label{tab:FR1.25 - Lokaal detailformulier toevoegen}
        \end{table}

%Lokaal detail formulier verwijderen        
\noindent\begin{table}[H]
            \begin{tabular}{l | p{10cm}}
                \textbf{ID:} & FR1.26 \\ \hline
                \textbf{TITEL:} & 5. Lokaal detail formulier verwijderen\\ \hline
                \textbf{PRIORITEIT:} &  hoog \\ \hline
                \textbf{PREREQUISITIES:} & \\ \hline
                \textbf{TOEGANG:} & Programmabeheerder \\ \hline
                \textbf{BESCHRIJVING:} & De Programmabeheerder moet in staat zijn lokalen te verwijderen (In het geval dat deze niet meer in gebruik worden genomen).\\ 
            \end{tabular}\\
            \caption{FR1.26 - Lokaal detail formulier verwijderen}
            \label{tab:FR1.26 - Lokaal detail formulier verwijderen}
        \end{table}
        
\noindent\begin{table}[H]
            \begin{tabular}{l | p{10cm}}
                \textbf{ID:} & FR1.27 \\ \hline
                \textbf{TITEL:} & Lokaal detail formulier aanpassen\\ \hline
                \textbf{PRIORITEIT:} &  medium \\ \hline
                \textbf{PREREQUISITIES:} & \\ \hline
                \textbf{TOEGANG:} & Programmabeheerder \\ \hline
                \textbf{BESCHRIJVING:} & Mocht er nood zijn aan het aanpassen van een lokaal formulier, moet de Programmabeheerder de optie hebben om dit te doen.\\ 
            \end{tabular}\\
            \caption{FR1.27 - Lokaal detail formulier aanpassen}
            \label{tab:FR1.27 - Lokaal detail formulier aanpassen}
        \end{table}

\noindent\begin{table}[H]
            \begin{tabular}{l | p{10cm}}
                \textbf{ID:} & FR1.28 \\ \hline
                \textbf{TITEL:} & Lesprogramma aanmaken voor een richting\\ \hline
                \textbf{PRIORITEIT:} &  hoog \\ \hline
                \textbf{PREREQUISITIES:} & \\ \hline
                \textbf{TOEGANG:} & Programmabeheerder \\ \hline
                \textbf{BESCHRIJVING:} & De Programmabeheerder is in staat om een lesprogramma manueel aan te maken voor een richting. Deze vraagt dan aan de Scheduler om een lesprogramma op te stellen voor een aantal vakken die bij die richting horen.\\ 
            \end{tabular}\\
            \caption{FR1.28 - Lesprogramma aanmaken voor een richting}
            \label{tab:FR1.28 - Lesprogramma aanmaken voor een richting}
        \end{table}

\noindent\begin{table}[H]
            \begin{tabular}{l | p{10cm}}
                \textbf{ID:} & FR1.29 \\ \hline
                \textbf{TITEL:} & Bekijken lesserooster als specifiek gebruiker\\ \hline
                \textbf{PRIORITEIT:} &  laag \\ \hline
                \textbf{PREREQUISITIES:} & \\ \hline
                \textbf{TOEGANG:} & Programmabeheerder \\ \hline
                \textbf{BESCHRIJVING:} & De Programmabeheerder kan via een naam en rolnummer het lesrooster van een specifieke gebruiker bekijken. \\ 
            \end{tabular}\\
            \caption{FR1.29 - Bekijken lesserooster als specifiek gebruiker}
            \label{tab:FR1.29 - Bekijken lesserooster als specifiek gebruiker}
        \end{table}
        
\noindent\begin{table}[H]
            \begin{tabular}{l | p{10cm}}
                \textbf{ID:} & FR5.9 \\ \hline
                \textbf{TITEL:} & Bevriezen van een voorstel\\ \hline
                \textbf{PRIORITEIT:} &  medium \\ \hline
                \textbf{PREREQUISITIES:} & \\ \hline
                \textbf{TOEGANG:} & Programmabeheerder \\ \hline
                \textbf{BESCHRIJVING:} & Een voorstel van een professor kan worden worden bevroren en de als het schema word aangemaakt ligt dit voorstel al vast\\ 
            \end{tabular}\\
            \caption{FR1.30 - Bevriezen van een voorstel}
            \label{tab:FR1.30 - Bevriezen van een voorstel}
        \end{table}
        
\noindent\begin{table}[H]
            \begin{tabular}{l | p{10cm}}
                \textbf{ID:} & FR1.31 \\ \hline
                \textbf{TITEL:} & Bevriezen volledige planning\\ \hline
                \textbf{PRIORITEIT:} &  medium \\ \hline
                \textbf{PREREQUISITIES:} & \\ \hline
                \textbf{TOEGANG:} & Programmabeheerder \\ \hline
                \textbf{BESCHRIJVING:} & Op het moment dat de planning vast is kunnen studenten deze bekijken. Het moet echter nog mogelijk zijn om deze aan te passen en daarom is heeft de Programmabeheerder de optie om de volledige planning te bevriezen en onzichtbaar te maken voor studenten.\\ 
            \end{tabular}\\
            \caption{FR1.31 - Bevriezen volledige planning}
            \label{tab:FR1.31 - Bevriezen volledige planning}
        \end{table}   
\clearpage