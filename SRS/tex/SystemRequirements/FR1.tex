\subsection{User Management - FR1}
%Profiel Aanmaken
	\begin{table}[H]
    		\begin{tabular}{l | p{10cm}}
        \textbf{ID:} & FR1.1 \\ \hline
        \textbf{TITEL:} & Profiel aanmaken \\ \hline
        \textbf{PRIORITEIT:} &  Hoog \\ \hline
        \textbf{PREREQUISITIES:} & Geen\\ \hline
        \textbf{TOEGANG:} &  Gast \\ \hline
        \textbf{BESCHRIJVING:} & Gebruiker zal zijn persoonlijke informatie moeten ingeven:
                                    \begin{itemize}\itemsep1pt \parskip0pt \parsep0pt
                                        \item Naam
                                        \item Voornaam
                                        \item Gebruikersnaam
                                        \item Wachtwoord
                                        \item Geboortedatum
                                        \item E-mail
                                        \item Rolnummer
                                        \item Richting (= vaste lijst)
                                        \item Gewenste taal
                                    \end{itemize}\\
    \end{tabular} 
    \caption{FR1.1 - Profiel aanmaken}
    \label{tab:FR1.1 -Profiel aanmaken}
\end{table}

\textbf{Stappenplan:}
\begin{enumerate}
\item De niet-aangemelde gebruiker (voortaan gebruiker geheten) ziet het beginvenster van de applicatie in zijn webbrowser.
\item De gebruiker klikt op de aanwezige knop om een profiel aan te maken.
\item De gebruiker krijgt een formulier te zien met invulbare velden voor zijn naam, voornaam, gebruikersnaam, wachtwoord, geboortedatum, email. Tevens vindt hij in het formulier 2 lijsten, respectievelijk voor zijn richting en gewenste taal te selecteren.
\item De gebruiker vult al de gegevens en selecteert de juiste optie in elke lijst.
\item Nadat de gebruiker deze gegevens volledig heeft ingevuld, klikt hij op de knop om de aanmaak van het profiel te beëindigen.
\item De gebruiker krijgt een melding op het scherm te zien dat het aanmaken van het profiel voltooid is.
\item Het systeem stuurt een email naar het opgegeven emailadres ter bevestiging.
\end{enumerate}

%TODO
\textbf{Uitzonderingen:}

%Aanmelden
\noindent\begin{table}[H]
            \begin{tabular}{l | p{10cm}}
                \textbf{ID:} & FR1.2 \\ \hline
                \textbf{TITEL:} & Aanmelden \\ \hline
                \textbf{PRIORITEIT:} &  Hoog \\ \hline
                \textbf{PREREQUISITIES:} & Gebruiker moet profiel bezitten\\ \hline
                \textbf{TOEGANG:} &  Gast \\ \hline
                \textbf{BESCHRIJVING:} & Gebruiker vult zijn username en wachtwoord in en klikt op “aanmelden”\\
            \end{tabular}
            \caption{FR1.2 - Aanmelden}
            \label{tab:FR1.2 - Aanmelden}
        \end{table}
        
\textbf{Stappenplan:}
\begin{enumerate}
\item De niet-aangemelde gebruiker (voortaan gebruiker geheten) ziet het beginvenster van de applicatie in zijn webbrowser. Hier ziet de gebruiker een formulier met 2 velden om zijn gebruikersnaam en wachtwoord in te typen met daaronder een knop om in te loggen
\item De gebruiker vult zijn gebruikersnaam en wachtwoord in.
\item De gebruiker klikt op de knop om in te loggen.
\item De gebruiker is ingelogd en ziet nu zijn persoonlijk lessenrooster.
\end{enumerate}

%TODO
\textbf{Uitzonderingen:}
\begin{itemize}
\item Ingevoerde gebruikersnaam is niet gevonden.
\item Ingevoerd wachtwoord is verkeerd.
\end{itemize}


%Wachtwoord vergeten
\noindent\begin{table}[H]
            \begin{tabular}{l | p{10cm}}
                \textbf{ID:} & FR1.3 \\ \hline
                \textbf{TITEL:} & Wachtwoord vergeten \\ \hline
                \textbf{PRIORITEIT:} &  Medium \\ \hline
                \textbf{PREREQUISITIES:} & Gebruiker moet profiel bezitten\\ \hline
                \textbf{TOEGANG:} &  Gast \\ \hline
                \textbf{BESCHRIJVING:} & Mocht een gebruiker zijn wachtwoord vergeten, dan moet deze e-mail invoeren. 
                                        Hierna zal een e-mail met een reset link verstuurd worden naar het ingevoerde e-mail adress. 
                                        Als e-mail adress niet bestaat dient er een error bericht gegeven te worden aan de gebruiker\\
            \end{tabular}\\
            \caption{FR1.3 - Wachtwoord vergeten}
            \label{tab:FR1.3 - Wachtwoord vergeten}
        \end{table}

\textbf{Stappenplan:}
\begin{enumerate}
\item De niet-aangemelde gebruiker (voortaan gebruiker geheten) ziet het beginvenster van de applicatie in zijn webbrowser. Hier ziet de gebruiker een knop met de tekst "Wachtwoord vergeten?".
\item De gebruiker klikt op deze knop. Er verschijnt een nieuw venster op het scherm, met de vraag om het mailadres van zijn profiel in te vullen in het formulier op hetzelfde scherm.
\item Het systeem verstuurt een email naar het opgegeven mailadres met als inhoud een "resetlink", een speciale URL om het wachtwoord tijdelijk te kunnen wijzigen.
\item De gebruiker opent de "resetlink" in de e-mail en krijgt een venster te zien in zijn webbrowser met een formulier om het nieuwe wachtwoord in te voeren en deze te bevestigen.
\item De gebruiker vult het nieuwe wachtwoord in en bevestigd deze door ze op exact dezelfde schrijfwijze in te voeren in het tweede veld van het formulier.
\item De gebruiker klikt op de knop met de naam "Wachtwoord wijzigen" om de wijziging door te voeren.
\item De gebruiker krijgt een melding op het scherm dat het wachtwoord is gewijzigd en wordt terug naar het beginscherm van de applicatie geleid.
\end{enumerate}

%TODO
\textbf{Uitzonderingen:}
\begin{itemize}
\item Mailadres bestaat niet.
\item Wachtwoord voldoet niet aan de minimale vereisten.
\item Wachtwoord is foutief bevestigd.
\end{itemize}

%Ingeschreven vakken aanpassen
\noindent\begin{table}[H]
            \begin{tabular}{l | p{10cm}}
                \textbf{ID:} & FR1.8 \\ \hline
                \textbf{TITEL:} & Ingeschreven vakken aanpassen\\ \hline
                \textbf{PRIORITEIT:} &  High \\ \hline
                \textbf{PREREQUISITIES:} & \\ \hline
                \textbf{TOEGANG:} &  Student \\ \hline
                \textbf{BESCHRIJVING:} & Een aangemelde student kan de vakken waarvoor deze is ingeschreven aanpassen (zolang de VUB dit toelaat!). 
                                        De student kan zich uitschrijven voor een vak door te klikken op de knop getiteld “Uitschrijven” die zich naast het vak bevind waarvoor deze zich wil uitschrijven. 
                                        Er zal dan een window geopend worden die nog vraagt of de gebruiker zeker is, die uitgerust is met een “ja” en “nee” knop. 
                                        Om zich in te schrijven voor een vak moet de student een vak opzoeken. 
                                        Hiervoor moet eerst de richting opgegeven waar het vak toe behoort en dan kan de gebruiker in de lijst van vakken zoeken op naam en vaknummer en wordt dan naar een pagina doorgestuurd waarop deze in lijstvorm te zien zijn.\\
            \end{tabular}\\
            \caption{FR1.8 - Ingeschreven vakken aanpassen}
            \label{tab:FR1.8 - Ingeschreven vakken aanpassen}
        \end{table}
        
%Taal Aanpassen
\noindent\begin{table}[H]
            \begin{tabular}{l | p{10cm}}
                \textbf{ID:} & FR1.10 \\ \hline
                \textbf{TITEL:} & Taal aanpassen\\ \hline
                \textbf{PRIORITEIT:} &  Laag \\ \hline
                \textbf{PREREQUISITIES:} & \\ \hline
                \textbf{TOEGANG:} &  Student, Assistent, Professor \\ \hline
                \textbf{BESCHRIJVING:} & Een aangemelde gebruiker kan zijn taal van voorkeur aanpassen. 
                                        Hiervoor drukt de gebruiker op de knop getiteld “Taal”, hierna zal er een keuze zijn aan talen waarin “CalZone” beschikbaar is.\\ 
            \end{tabular}\\
            \caption{FR1.10 - Taal aanpassen}
            \label{tab:FR1.10 - Taal aanpassen}
        \end{table}

%Afmelden
\noindent\begin{table}[H]
            \begin{tabular}{l | p{10cm}}
                \textbf{ID:} & FR1.11 \\ \hline
                \textbf{TITEL:} & Afmelden\\ \hline
                \textbf{PRIORITEIT:} &  Hoog \\ \hline
                \textbf{PREREQUISITIES:} & \\ \hline
                \textbf{TOEGANG:} &  Student, Assistent, Professor, Programmabeheerder \\ \hline
                \textbf{BESCHRIJVING:} & Om af te melden moet een gebruiker simpelweg op de knop getiteld “Afmelden” drukken die de gebruiker terug naar de inlog pagina van het programma brengt. Geldig voor alle gebruikers\\
            \end{tabular}\\
            \caption{FR1.11 - Afmelden}
            \label{tab:FR1.11 - Afmelden}
        \end{table}

%Profiel gegevens aanpassen
\noindent\begin{table}[H]
            \begin{tabular}{l | p{10cm}}
                \textbf{ID:} & FR1.12 \\ \hline
                \textbf{TITEL:} & Profiel gegevens aanpassen\\ \hline
                \textbf{PRIORITEIT:} &  medium \\ \hline
                \textbf{PREREQUISITIES:} & \\ \hline
                \textbf{TOEGANG:} & Student, Assistent, Professor \\ \hline
                \textbf{BESCHRIJVING:} & Een gebruiker kan zijn profielgegevens aanpassen door te drukken op de knop getiteld “Profiel”. Geldig voor alle gebruikers van het systeem. 
                                        En kan dan de volgende informatie aanpassen:
                                        \begin{itemize}\itemsep1pt \parskip0pt \parsep0pt
                                        \item Naam
                                        \item Voornaam
                                        \item Gebruikersnaam
                                        \item Wachtwoord
                                        \item Geboortedatum
                                        \item E-mail
                                        \item Rolnummer
                                        \item Richting (= vaste lijst)
                                        \end{itemize}\\
            \end{tabular}\\
            \caption{FR1.12 - Profiel gegevens aanpassen}
            \label{tab:FR1.12 - Profiel gegevens aanpassen}
        \end{table} 
               
        
%Beschikbaarheids formulier aanmaken   
\noindent\begin{table}[H]
            \begin{tabular}{l | p{10cm}}
                \textbf{ID:} & FR1.16 \\ \hline
                \textbf{TITEL:} & Beschikbaarheidsformulier aanmaken\\ \hline
                \textbf{PRIORITEIT:} &  laag \\ \hline
                \textbf{PREREQUISITIES:} & \\ \hline
                \textbf{TOEGANG:} & Assistent, Professor \\ \hline
                \textbf{BESCHRIJVING:} & Een gebruiker (van type toegang) heeft de optie om een formulier in te dienen waarin hij aangeeft wat zijn vrije momenten zijn om les te geven, de Scheduler houd hier dan rekening mee bij het opstellen van de lesroosters.\\ 
            \end{tabular}\\
            \caption{FR1.16 - Beschikbaarheidsformulier aanmaken}
            \label{tab:FR1.16 - Beschikbaarheidsformulier aanmaken}
        \end{table}

%Beschikbaarheids fromulier aanpassen
\noindent\begin{table}[H]
            \begin{tabular}{l | p{10cm}}
                \textbf{ID:} & FR1.17 \\ \hline
                \textbf{TITEL:} & Beschikbaarheids formulier aanpassen\\ \hline
                \textbf{PRIORITEIT:} &  laag \\ \hline
                \textbf{PREREQUISITIES:} & Gebruiker (van type toegang) moet al een beschikbaarheids formulier ingedient hebben\\ \hline
                \textbf{TOEGANG:} & Assistent, Professor \\ \hline
                \textbf{BESCHRIJVING:} & Een gebruiker (van type toegang) heeft de optie om zijn beschikbaarheids formulier aan te passen.\\ 
            \end{tabular}\\
            \caption{FR1.17 - Beschikbaarheidsformulier aanpassen}
            \label{tab:FR1.16 - Beschikbaarheids formulier aanpassen}
        \end{table}

        
\clearpage
