\subsection{Niet-aangemelde gebruiker - FR1}

\begin{table}[h]
    \begin{tabular}{l | p{10cm}}
        \textbf{ID:} & FR1.1 \\ \hline
        \textbf{TITEL:} & Profiel aanmaken \\ \hline
        \textbf{PRIORITEIT:} &  Hoog \\ \hline
        \textbf{PREREQUISITIES:} & Geen\\ \hline
        \textbf{BESCHRIJVING:} & Gebruiker zal zijn persoonlijke informatie moeten ingeven:
                                    \begin{itemize}\itemsep1pt \parskip0pt \parsep0pt
                                        \item Naam
                                        \item Voornaam
                                        \item Gebruikersnaam
                                        \item Wachtwoord
                                        \item Geboortedatum
                                        \item E-mail
                                        \item Rolnummer
                                        \item Richting (= vaste lijst)
                                        \item Gewenste taal
                                    \end{itemize}\\
    \end{tabular} 
    \caption{FR1.1}
    \label{tab:myfirsttable}
\end{table}

%\rule{\textwidth}{.4pt}\\

\noindent\begin{table}[h]
            \begin{tabular}{l | p{10cm}}
                \textbf{ID:} & FR1.2 \\ \hline
                \textbf{TITEL:} & Aanmelden \\ \hline
                \textbf{PRIORITEIT:} &  Hoog \\ \hline
                \textbf{PREREQUISITIES:} & Gebruiker moet profiel bezitten\\ \hline
                \textbf{BESCHRIJVING:} & Gebruiker vult zijn username en wachtwoord in en klikt op “aanmelden”\\
            \end{tabular}
            \caption{FR1.2}
            \label{tab:mysecondtable}
        \end{table}

%\rule{\textwidth}{.4pt}\\

\noindent\begin{table}[h]
            \begin{tabular}{l | p{10cm}}
                \textbf{ID:} & FR1.3 \\ \hline
                \textbf{TITEL:} & Wachtwoord vergeten \\ \hline
                \textbf{PRIORITEIT:} &  Medium \\ \hline
                \textbf{PREREQUISITIES:} & Gebruiker moet profiel bezitten\\ \hline
                \textbf{BESCHRIJVING:} & Mocht een gebruiker zijn wachtwoord vergeten, dan moet deze e-mail invoeren. 
                                        Hierna zal een e-mail met een reset link verstuurd worden naar het ingevoerde e-mail adress. 
                                        Als e-mail adress niet bestaat dient er een error bericht gegeven te worden aan de gebruiker\\
            \end{tabular}\\
            \caption{FR1.3}
            \label{tab:mythirdtable}
        \end{table}

%\rule{\textwidth}{.4pt}\\

\noindent\begin{table}[h]
            \begin{tabular}{l | p{10cm}}
                \textbf{ID:} & FR1.4 \\ \hline
                \textbf{TITEL:} & Lesrooster van richting bekijken \\ \hline
                \textbf{PRIORITEIT:} &  High \\ \hline
                \textbf{PREREQUISITIES:} & \\ \hline
                \textbf{BESCHRIJVING:} & Een niet ingelogde gebruiker heeft de optie om het standaard lesrooster van een bepaalde richting te bekijken. 
                                        Hiervoor moet hij een richting en jaar uitkiezen.\\
            \end{tabular}\\
            \caption{FR1.4}
            \label{tab:myfourthtable}
        \end{table}

%\rule{\textwidth}{.4pt}\\

\noindent\begin{table}[h]
            \begin{tabular}{l | p{10cm}}
                \textbf{ID:} & FR1.5 \\ \hline
                \textbf{TITEL:} & Lesrooster van specifiek vak bekijken \\ \hline
                \textbf{PRIORITEIT:} &  High \\ \hline
                \textbf{PREREQUISITIES:} & \\ \hline
                \textbf{BESCHRIJVING:} & Een niet-ingelogde gebruiker heeft de optie om het lesrooster van een bepaald vak te bekijken. 
                                        Hiervoor moet hij \'{e}\'{e} van de bestaande vakken aan de VUB kiezen.\\
            \end{tabular}\\
            \caption{FR1.5}
            \label{tab:myfifthtable}
        \end{table}



%\rule{\textwidth}{.4pt}\\

\noindent\begin{table}[h]
            \begin{tabular}{l | p{10cm}}
                \textbf{ID:} & FR1.6 \\ \hline
                \textbf{TITEL:} & Lesrooster van een lokaal bekijken\\ \hline
                \textbf{PRIORITEIT:} &  High \\ \hline
                \textbf{PREREQUISITIES:} & \\ \hline
                \textbf{BESCHRIJVING:} & Een niet ingelogde gebruiker heeft de optie om het lesrooster van een leslokaal te bekijken, d.w.z. wanneer dat lokaal bezet is. 
                                        Hiervoor moet de gebruiker zelf een les lokaal invullen van de vorm G.V.L (met G = gebouw, V = verdieping en L = lokaal)\\
            \end{tabular}\\
            \caption{FR1.6}
            \label{tab:mysixthtable}
        \end{table}
\clearpage
%\rule{\textwidth}{.4pt}\\