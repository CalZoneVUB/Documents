\subsection{Aangemelde programmabeheerder - FR5}

\noindent\begin{table}[h]
            \begin{tabular}{l | p{10cm}}
                \textbf{ID:} & FR5.1 \\ \hline
                \textbf{TITEL:} & Vakken toevoegen op basis van ingediend detail formulier\\ \hline
                \textbf{PRIORITEIT:} &  hoog \\ \hline
                \textbf{PREREQUISITIES:} & \\ \hline
                \textbf{BESCHRIJVING:} & De Programmabeheerder kan vakken toevoegen aan de database met behulp van vak formulieren die ingediend worden door professoren. \\ 
            \end{tabular}\\
            \caption{FR5.1}
            \label{tab:myfifteenthtable}
        \end{table}
        
\noindent\begin{table}[h]
            \begin{tabular}{l | p{10cm}}
                \textbf{ID:} & FR5.2 \\ \hline
                \textbf{TITEL:} & Vakken verwijderen van lesprogramma\\ \hline
                \textbf{PRIORITEIT:} &  hoog \\ \hline
                \textbf{PREREQUISITIES:} & \\ \hline
                \textbf{BESCHRIJVING:} & De Programmabeheerder kan ook vakken verwijderen uit lesprogramma’s \\ 
            \end{tabular}\\
            \caption{FR5.2}
            \label{tab:myfifteenthtable}
        \end{table}
        
\noindent\begin{table}[h]
            \begin{tabular}{l | p{10cm}}
                \textbf{ID:} & FR5.3 \\ \hline
                \textbf{TITEL:} & Aanpassen detail formulier specifiek vak\\ \hline
                \textbf{PRIORITEIT:} &  medium \\ \hline
                \textbf{PREREQUISITIES:} & \\ \hline
                \textbf{BESCHRIJVING:} & De Programmabeheerder kan net zoals een professor aanpassingen toebrengen op een vak formulier. \\ 
            \end{tabular}\\
            \caption{FR5.3}
            \label{tab:myfifteenthtable}
        \end{table}
        
\noindent\begin{table}[h]
            \begin{tabular}{l | p{10cm}}
                \textbf{ID:} & FR5.4 \\ \hline
                \textbf{TITEL:} & Lokaal detail formulier toevoegen\\ \hline
                \textbf{PRIORITEIT:} &  hoog \\ \hline
                \textbf{PREREQUISITIES:} & \\ \hline
                \textbf{BESCHRIJVING:} & De Programmabeheerder moet in staat zijn om lokalen toe te voegen aan de databse zodat de Scheduler die kan gebruiken in zijn planning. Lokalen hebben een ID nummer van de vorm G.V.L (met G = gebouw, V = verdieping en L = lokaal) en hebben ook een maximaal aantal plaats.\\ 
            \end{tabular}\\
            \caption{FR5.4}
            \label{tab:myfifteenthtable}
        \end{table}
        
\noindent\begin{table}[h]
            \begin{tabular}{l | p{10cm}}
                \textbf{ID:} & FR5.5 \\ \hline
                \textbf{TITEL:} & 5. Lokaal detail formulier verwijderen\\ \hline
                \textbf{PRIORITEIT:} &  hoog \\ \hline
                \textbf{PREREQUISITIES:} & \\ \hline
                \textbf{BESCHRIJVING:} & De Programmabeheerder moet in staat zijn lokalen te verwijderen (In het geval dat deze niet meer in gebruik worden genomen).\\ 
            \end{tabular}\\
            \caption{FR5.5}
            \label{tab:myfifteenthtable}
        \end{table}
        
\noindent\begin{table}[h]
            \begin{tabular}{l | p{10cm}}
                \textbf{ID:} & FR5.6 \\ \hline
                \textbf{TITEL:} & 5. Lokaal detail formulier aanpassen\\ \hline
                \textbf{PRIORITEIT:} &  medium \\ \hline
                \textbf{PREREQUISITIES:} & \\ \hline
                \textbf{BESCHRIJVING:} & Mocht er nood zijn aan het aanpassen van een lokaal formulier, moet de Programmabeheerder de optie hebben om dit te doen.\\ 
            \end{tabular}\\
            \caption{FR5.6}
            \label{tab:myfifteenthtable}
        \end{table}

\noindent\begin{table}[h]
            \begin{tabular}{l | p{10cm}}
                \textbf{ID:} & FR5.7 \\ \hline
                \textbf{TITEL:} & 5. Lesprogramma aanmaken voor een richting\\ \hline
                \textbf{PRIORITEIT:} &  hoog \\ \hline
                \textbf{PREREQUISITIES:} & \\ \hline
                \textbf{BESCHRIJVING:} & De Programmabeheerder is in staat om een lesprogramma manueel aan te maken voor een richting. Deze vraagt dan aan de Scheduler om een lesprogramma op te stellen voor een aantal vakken die bij die richting horen.\\ 
            \end{tabular}\\
            \caption{FR5.7}
            \label{tab:myfifteenthtable}
        \end{table}

\noindent\begin{table}[h]
            \begin{tabular}{l | p{10cm}}
                \textbf{ID:} & FR5.8 \\ \hline
                \textbf{TITEL:} & 5. Bekijken lesserooster als specifiek gebruiker\\ \hline
                \textbf{PRIORITEIT:} &  laag \\ \hline
                \textbf{PREREQUISITIES:} & \\ \hline
                \textbf{BESCHRIJVING:} & De Programmabeheerder kan via een naam en rolnummer het lesrooster van een specifieke gebruiker bekijken. \\ 
            \end{tabular}\\
            \caption{FR5.8}
            \label{tab:myfifteenthtable}
        \end{table}
        
\noindent\begin{table}[h]
            \begin{tabular}{l | p{10cm}}
                \textbf{ID:} & FR5.9 \\ \hline
                \textbf{TITEL:} & 5. Bevriezen van een voorstel\\ \hline
                \textbf{PRIORITEIT:} &  medium \\ \hline
                \textbf{PREREQUISITIES:} & \\ \hline
                \textbf{BESCHRIJVING:} & \\ 
            \end{tabular}\\
            \caption{FR5.9}
            \label{tab:myfifteenthtable}
        \end{table}
        
\noindent\begin{table}[h]
            \begin{tabular}{l | p{10cm}}
                \textbf{ID:} & FR5.10 \\ \hline
                \textbf{TITEL:} & 5. Bevriezen volledige planning\\ \hline
                \textbf{PRIORITEIT:} &  medium \\ \hline
                \textbf{PREREQUISITIES:} & \\ \hline
                \textbf{BESCHRIJVING:} & Op het moment dat de planning vast is kunnen studenten deze bekijken. Het moet echter nog mogelijk zijn om deze aan te passen en daarom is heeft de Programmabeheerder de optie om de volledige planning te bevriezen en onzichtbaar te maken voor studenten.\\ 
            \end{tabular}\\
            \caption{FR5.10}
            \label{tab:myfifteenthtable}
        \end{table}


\clearpage