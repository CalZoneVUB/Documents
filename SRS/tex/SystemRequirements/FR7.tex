\subsection{Mobiliteit - FR7}

% Exporteren van kalender naar mobiele apparaten
\subsubsection{Exporteren Van Kalender Naar Mobiele Apparaten}
\noindent\begin{table}[H]
            \begin{tabular}{l | p{10cm}}
                \textbf{ID:} & FR7.1 \\ \hline
                \textbf{TITEL:} & Exporteren Van Kalender Naar Mobiele Apparaten\\ \hline
                \textbf{PRIORITEIT:} &  Medium \\ \hline
                \textbf{PREREQUISITIES:} & \\ \hline
                \textbf{TOEGANG:} & Aangemelde Gebruiker \\ \hline
                \textbf{BESCHRIJVING:} & De applicatie moet toelaten om kalenders te importeren naar mobiele apparaten zoals smartphones.\\ 
            \end{tabular}\\
            \caption{FR7.1 - Exporteren Van Kalender Naar Mobiele Apparaten}
            \label{tab:FR7.1 - Exporteren Van Kalender Naar Mobiele Apparaten}
        \end{table}
        
%GPS locatie van lokaal gebruiken om lessenrooster op te halen
\subsubsection{GPS Lokaal Identificatie} 
\noindent\begin{table}[H]
            \begin{tabular}{l | p{10cm}}
                \textbf{ID:} & FR7.2 \\ \hline
                \textbf{TITEL:} & GPS Lokaal Identificatie\\ \hline
                \textbf{PRIORITEIT:} &  Laag \\ \hline
                \textbf{PREREQUISITIES:} & \\ \hline
                \textbf{TOEGANG:} & Aangemelde Gebruiker \\ \hline
                \textbf{BESCHRIJVING:} & Met een mobiel apparaat is het mogelijk om samen met de GPS locatie van een lokaal, het lessenrooster van dat lokaal op te vragen.\\ 
            \end{tabular}\\
            \caption{FR7.2 - GPS Lokaal Identificatie}
            \label{tab:FR7.2 - GPS Lokaal Identificatie}
        \end{table}


%Check-in button voor lokaal
\subsubsection{Check-in Functionaliteit} 
\noindent\begin{table}[H]
            \begin{tabular}{l | p{10cm}}
                \textbf{ID:} & FR7.3 \\ \hline
                \textbf{TITEL:} & Check-in Functionaliteit\\ \hline
                \textbf{PRIORITEIT:} &  Laag \\ \hline
                \textbf{PREREQUISITIES:} & \\ \hline
                \textbf{TOEGANG:} & Aangemelde Gebruiker \\ \hline
                \textbf{BESCHRIJVING:} & Een gebruiker kan aangeven dat deze zich in een les bevind door op een "check-in" knop te drukken. Deze functionaliteit word geïmplementeerd om later mogelijke gamification technieken in de applicatie te implementeren.\\ 
            \end{tabular}\\
            \caption{FR7.3 - Check-in Functionaliteit}
            \label{tab:FR7.3 - Check-in Functionaliteit}
        \end{table}

%Push-notifiactions
\subsubsection{Push Notificaties} 
\noindent\begin{table}[H]
            \begin{tabular}{l | p{10cm}}
                \textbf{ID:} & FR7.4 \\ \hline
                \textbf{TITEL:} & Push Notificaties\\ \hline
                \textbf{PRIORITEIT:} &  Laag \\ \hline
                \textbf{PREREQUISITIES:} & \\ \hline
                \textbf{TOEGANG:} & Aangemelde Gebruiker \\ \hline
                \textbf{BESCHRIJVING:} & Een aangemelde gebruiker kan om zijn mobiele apparaten notificaties krijgen over zijn vakken (bijvoorbeeld dat deze van lokaal verandert is). \\ 
            \end{tabular}\\
            \caption{FR7.4 - Push Notificaties}
            \label{tab:FR7.4 - Push Notificaties}
        \end{table}
\clearpage

      
