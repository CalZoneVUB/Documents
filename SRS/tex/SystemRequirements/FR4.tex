\subsection{Aangemelde professor - FR4}

\noindent\begin{table}[h]
            \begin{tabular}{l | p{10cm}}
                \textbf{ID:} & FR4.1 \\ \hline
                \textbf{TITEL:} & Taal aanpassen\\ \hline
                \textbf{PRIORITEIT:} &  Laag \\ \hline
                \textbf{PREREQUISITIES:} & \\ \hline
                \textbf{BESCHRIJVING:} & Een aangemelde professor kan zijn taal van voorkeur aanpassen. 
                                        Hiervoor drukt de gebruiker op de knop getiteld “Taal”, hierna zal er een keuze zijn aan talen waarin “CalZone” beschikbaar is.\\ 
            \end{tabular}\\
            \caption{FR4.1}
            \label{tab:FR4.1}
        \end{table}
        
\noindent\begin{table}[h]
            \begin{tabular}{l | p{10cm}}
                \textbf{ID:} & FR4.2 \\ \hline
                \textbf{TITEL:} & Last-minute tijdstip les aanpasen\\ \hline
                \textbf{PRIORITEIT:} &  medium \\ \hline
                \textbf{PREREQUISITIES:} & \\ \hline
                \textbf{BESCHRIJVING:} & Een professor heeft de optie om het tijdstip van een les last-minute aan te passen. De professor drukt dan op de knop getiteld “Lesuur aanpassen” en wordt dan gevraagd om een nieuwe dag en een tijdstip uit te kiezen. Hierna wordt dan een e-mail gestuurd naar alle studenten die dat vak volgen met het nieuwe tijdstip van het vak. Dit mag geen conflicten veroorzaken met verplichte vakken en mag enkel eenmalig gebeuren. \\ 
            \end{tabular}\\
            \caption{FR4.2}
            \label{tab:FR4.2}
        \end{table}
        
\noindent\begin{table}[h]
            \begin{tabular}{l | p{10cm}}
                \textbf{ID:} & FR4.3 \\ \hline
                \textbf{TITEL:} & Locatie les aanpasen\\ \hline
                \textbf{PRIORITEIT:} &  medium \\ \hline
                \textbf{PREREQUISITIES:} & \\ \hline
                \textbf{BESCHRIJVING:} & Een professor heeft de optie om de locatie van een vak aan te passen. 
                                        De professor drukt dan op de knop getiteld “Lokaal aanpassen” en wordt dan gevraagd om een nieuwe locatie uit te kiezen, deze locatie is dan van de vorm G.V.L (met G = gebouw, V = verdieping en L = lokaal). De Scheduler zal dan nagaan of het ingevoerde lokaal vrij is op het tijdstip van het vak. Keurt de Scheduler het nieuwe lokaal goed, dan wordt er een e-mail gestuurd naar alle studenten die het vak volgen met daarin het nieuwe lokaal van het vak. Keurt de Scheduler het nieuwe lokaal af, dan wordt de prefessor gevraagd een ander lokaal in te geven.\\ 
            \end{tabular}\\
            \caption{FR4.3}
            \label{tab:FR4.3}
        \end{table}
        
\noindent\begin{table}[h]
            \begin{tabular}{l | p{10cm}}
                \textbf{ID:} & FR4.4 \\ \hline
                \textbf{TITEL:} & Details les aanpassen\\ \hline
                \textbf{PRIORITEIT:} &  laag \\ \hline
                \textbf{PREREQUISITIES:} & \\ \hline
                \textbf{BESCHRIJVING:} & Een professor kan de beschrijving en details van een vak aanpassen.                                              Hiervoor zal er een knop getiteld “Vak details aanpassen” zijn waarna de professor vrij is in het aanpassen van de details. \\ 
            \end{tabular}\\
            \caption{FR4.4}
            \label{tab:FR4.4}
        \end{table}
        
\noindent\begin{table}[h]
            \begin{tabular}{l | p{10cm}}
                \textbf{ID:} & FR4.5 \\ \hline
                \textbf{TITEL:} & Schema van lesgever bekijken\\ \hline
                \textbf{PRIORITEIT:} &  Medium \\ \hline
                \textbf{PREREQUISITIES:} & \\ \hline
                \textbf{BESCHRIJVING:} & Een professor kan zijn eigen schema van alle lessen bekijken. Dit gaat over verschillende vakken. Optioneel zou hij ook kunnen selecteren om het schema op te vragen van al zijn lessen (inclusief WPO's gegeven door assistenten) \\ 
            \end{tabular}\\
            \caption{FR4.5}
            \label{tab:FR4.5}
        \end{table}
        
\noindent\begin{table}[h]
            \begin{tabular}{l | p{10cm}} 
                \textbf{ID:} & FR4.6 \\ \hline
                \textbf{TITEL:} & Detail formulier nieuw vak indienen\\ \hline
                \textbf{PRIORITEIT:} &  laag \\ \hline
                \textbf{PREREQUISITIES:} & \\ \hline
                \textbf{BESCHRIJVING:} & Mocht er nood zijn aan het toevoegen van een nieuw vak aan de database, dan kan een professor hiervoor een formulier opstellen met daarin:
        \begin{itemize}\itemsep1pt \parskip0pt \parsep0pt
                                        \item beschrijving van het vak
                                        \item aantal studiepunten
                                        \item nodige prerequisities
                                        \item welk semester het vak gegeven wordt
                                        \item aantal uren studie tijd
                                        \item studiegidsnummer
                                        \item 2e zittijd mogelijk
                                        \item onderwijstaal
                                        \item faculteit
                                        \item verantwoordelijke vakgroep
                                        \item onderwijsteam
                                        \item onderdelen en contacturen
                                        \item studiemateriaal
                                        \item leerresultaten
                                        \item beoordelingsinformatie
                                        \item Aanvullende info m.b.t. examinering
                                        \item Academische context
                                        \end{itemize}
                                        Het formulier wordt dan gestuurd naar de Programmabeheerder die het vak zal toevoegen. 
            \end{tabular}\\
            \caption{FR4.6}
            \label{tab:FR4.6}
        \end{table}
        
\noindent\begin{table}[h]
            \begin{tabular}{l | p{10cm}}
                \textbf{ID:} & FR4.7 \\ \hline
                \textbf{TITEL:} & Aanpassen detail formulier eigen vak \\ \hline
                \textbf{PRIORITEIT:} &  laag \\ \hline
                \textbf{PREREQUISITIES:} & De professor moet al een detail formulier ingedient hebben\\ \hline
                \textbf{BESCHRIJVING:} & Zolang het nieuwe vak nog niet toegevoegd is door de Programmabeheerder kan de professor nog aanpassingen aanbrengen op het ingediende formulier\\ 
            \end{tabular}\\
            \caption{FR4.7}
            \label{tab:FR4.7}
        \end{table}
        
\noindent\begin{table}[h]
            \begin{tabular}{l | p{10cm}}
                \textbf{ID:} & FR4.8 \\ \hline
                \textbf{TITEL:} & Beschikbaarheids formulier aanmaken \\ \hline
                \textbf{PRIORITEIT:} &  Medium \\ \hline
                \textbf{PREREQUISITIES:} & \\ \hline
                \textbf{BESCHRIJVING:} & 
                Invlullen van formulier met alle data voor inschrijven. Optie om steeds nieuwe entry to te voegen per week in deze vorm. Dag van de week / Van uur / Tot uur. Opties moet ook bestaan voor verschillende formulieren in te dienen per semester.\\
            \end{tabular}\\
            \caption{FR4.8}
            \label{tab:FR4.8}
        \end{table}
        
\noindent\begin{table}[h]
            \begin{tabular}{l | p{10cm}}
                \textbf{ID:} & FR4.9 \\ \hline
                \textbf{TITEL:} & Beschikbaarheids formulier aanpassen \\ \hline
                \textbf{PRIORITEIT:} &  laag \\ \hline
                \textbf{PREREQUISITIES:} & \\ \hline
                \textbf{BESCHRIJVING:} & Een professor heeft de optie om zijn beschikbaarheids formulier aan te passen. Dit is enkel mogelijk voor schemas vastliggen voor die bepaalde periode. Beschikbaarheids formulieren hebben de opties om te verschillen voor elk semester\\ 
            \end{tabular}\\
            \caption{FR4.9}
            \label{tab:FR4.9}
        \end{table}
        
\noindent\begin{table}[h]
            \begin{tabular}{l | p{10cm}}
                \textbf{ID:} & FR4.10 \\ \hline
                \textbf{TITEL:} & Afmelden\\ \hline
                \textbf{PRIORITEIT:} &  Hoog \\ \hline
                \textbf{PREREQUISITIES:} & \\ \hline
                \textbf{BESCHRIJVING:} & Om af te melden moet een gebruiker simpelweg op de knop getiteld “Afmelden” drukken die de gebruiker terug naar de inlog pagina van het programma brengt.\\
            \end{tabular}\\
            \caption{FR4.10}
            \label{tab:FR4.10}
        \end{table}
        
\noindent\begin{table}[h]
            \begin{tabular}{l | p{10cm}}
                \textbf{ID:} & FR4.11 \\ \hline
                \textbf{TITEL:} & Aanpassen planning vak\\ \hline
                \textbf{PRIORITEIT:} &  Medium \\ \hline
                \textbf{PREREQUISITIES:} & \\ \hline
                \textbf{BESCHRIJVING:} & Het moet mogelijk  zijn een vak te versplaatsen in de algemene planning voor elke week. Dit kan gebeuren op een schema dat al vast ligt en eventueel al bezig is. Als een verplaatsing geen conflicten veroorzaakt gebeurd dit automatisch. Anders wordt een voorstel doorgeschoven aan de programmabeheerder om het goed te keuren.\\
            \end{tabular}\\
            \caption{FR4.11}
            \label{tab:FR4.11}
        \end{table}
        
\noindent\begin{table}[h]
            \begin{tabular}{l | p{10cm}}
                \textbf{ID:} & FR4.12 \\ \hline
                \textbf{TITEL:} & Planning markeren als voorstel\\ \hline
                \textbf{PRIORITEIT:} &  medium \\ \hline
                \textbf{PREREQUISITIES:} & \\ \hline
                \textbf{BESCHRIJVING:} & Een professor kan een voorstel indienen aan de Scheduler. 
                                        Deze geeft dan in wanneer hij zijn vak het liefst zou geven en dan wordt dit doorgegeven aan de Scheduler die dan rekening probeert te houden met de ingegeven constraint. Dit staat los van de beschikbaarheid van de professor en lokalen.  \\
            \end{tabular}\\
            \caption{FR4.12}
            \label{tab:FR4.12}
        \end{table}
        
\clearpage
        