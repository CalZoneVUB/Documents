%Aanpassen detail formulier vak      
\noindent\begin{table}[H]
            \begin{tabular}{l | p{10cm}}
                \textbf{ID:} & FR3.13 \\ \hline
                \textbf{TITEL:} & Aanpassen detail formulier vak \\ \hline
                \textbf{PRIORITEIT:} &  laag \\ \hline
                \textbf{PREREQUISITIES:} & Detailformulier vak moet al bestaan\\ \hline
                \textbf{TOEGANG:} & Professor\\ \hline
                \textbf{BESCHRIJVING:} & Zolang het nieuwe vak nog niet toegevoegd is door de programmabeheerder kan de professor nog aanpassingen aanbrengen op het ingediende formulier. Ook is het mogelijk om aan bestaande detailformulieren aanpassingen te doen. Dit mag enkel gebeuren voordat het schema is gemaakt.\\ 
            \end{tabular}\\
            \caption{FR3.13 - Aanpassen detail formulier vak}
            \label{tab:FR3.13 - Aanpassen detailformulier vak}
        \end{table}

\textbf{Stappenplan:}
\begin{enumerate}
\item De professor ziet zijn beginscherm op zijn webbrowser. Hier ziet hij een optie om zijn docerende vakken te bezichtigen.
\item De professor klikt op deze optie en krijgt een lijst van zijn vakken te zien in een nieuwe pagina.
\item De professor klikt een van zijn vakken aan en krijgt een detailoverzicht met de eigenschappen van het vak te zien in een nieuwe pagina.
\item De professor wijzigt de gegevens en ziet onderaan een knop met de naam "Wijzigingen opslaan".
\item De professor klikt op de knop en de gegevens worden permanent gewijzigd.
\end{enumerate}

%Aanpassen planning vak        
\noindent\begin{table}[H]
            \begin{tabular}{l | p{10cm}}
                \textbf{ID:} & FR3.14 \\ \hline
                \textbf{TITEL:} & Aanpassen planning vak\\ \hline
                \textbf{PRIORITEIT:} &  Medium \\ \hline
                \textbf{PREREQUISITIES:} & \\ \hline
                \textbf{TOEGANG:} & Professor \\ \hline
                \textbf{BESCHRIJVING:} & Het moet mogelijk zijn een vak te verplaatsen in de algemene planning voor elke week. Dit kan gebeuren op een schema dat reeds vast ligt en eventueel al bezig is. Als een verplaatsing geen conflicten veroorzaakt gebeurt dit automatisch. Anders wordt een voorstel doorgeschoven aan de programmabeheerder om het goed te keuren.\\
            \end{tabular}\\
            \caption{FR3.14 - Aanpassen planning vak}
            \label{tab:FR3.14 - Aanpassen planning vak}
        \end{table}

\textbf{Stappenplan:}
\begin{enumerate}
\item De professor ziet zijn persoonlijk rooster. Dit rooster bevat de planning waarop hij zijn vakken doceren zal.
\item De professor klikt een vak aan en krijgt de optie om de datum van het  hoorcollege of werkcollege te verplaatsen naar een datum en tijdstip die nog niet versteken is.
\item De professor wijzigt de datum en het tijdstip en bevestigt dit door de knop met de naam "Verplaatsen" aan te klikken.
\item Het vak is verplaatst en is nu op het gewijzigde moment te zien in het rooster.
\end{enumerate}

\textbf{Uitzonderingen:}
\begin{itemize}
\item De wijzigingen wordt niet automatisch doorgevoerd
\end{itemize}

\noindent\begin{table}[H]
            \begin{tabular}{l | p{10cm}}
                \textbf{ID:} & FR3.15 \\ \hline
                \textbf{TITEL:} & Planning markeren als voorstel\\ \hline
                \textbf{PRIORITEIT:} &  medium \\ \hline
                \textbf{PREREQUISITIES:} & \\ \hline
                \textbf{TOEGANG:} & Professor \\ \hline
                \textbf{BESCHRIJVING:} & Een professor kan een voorstel indienen aan de Scheduler. 
                                        Deze geeft dan in wanneer hij zijn vak het liefst zou geven en dan wordt dit doorgegeven aan de Scheduler die dan rekening probeert te houden met de ingegeven constraint. Dit staat los van de beschikbaarheid van de professor en lokalen.  \\
            \end{tabular}\\
            \caption{FR3.15 - Planning markeren als voorstel}
            \label{tab:FR3.15 - Planning markeren als voorstel}
        \end{table}

\textbf{Stappenplan:}
\begin{itemize}
\item De professor ziet zijn beginscherm in zijn webbrowser. 
\end{itemize}

%Vakken verwijderen van lesprogramma        
\noindent\begin{table}[H]
            \begin{tabular}{l | p{10cm}}
                \textbf{ID:} & FR3.16 \\ \hline
                \textbf{TITEL:} & Vakken verwijderen van lesprogramma\\ \hline
                \textbf{PRIORITEIT:} &  hoog \\ \hline
                \textbf{PREREQUISITIES:} & \\ \hline
                \textbf{TOEGANG:} & Programmabeheerder \\ \hline
                \textbf{BESCHRIJVING:} & De Programmabeheerder kan ook vakken verwijderen uit lesprogramma’s \\ 
            \end{tabular}\\
            \caption{FR3.16 - Vakken verwijderen van lesprogramma}
            \label{tab:FR3.16 - Vakken verwijderen van lesprogramma}
        \end{table}

\textbf{Stappenplan:}
   
%Lokaal detailformulier toevoegen  
\noindent\begin{table}[H]
            \begin{tabular}{l | p{10cm}}
                \textbf{ID:} & FR3.17 \\ \hline
                \textbf{TITEL:} & Lokaal detailformulier toevoegen\\ \hline
                \textbf{PRIORITEIT:} &  hoog \\ \hline
                \textbf{PREREQUISITIES:} & \\ \hline
                \textbf{TOEGANG:} & Programmabeheerder \\ \hline
                \textbf{BESCHRIJVING:} & De Programmabeheerder moet in staat zijn om lokalen toe te voegen aan de databse zodat de Scheduler die kan gebruiken in zijn planning. Lokalen hebben een ID nummer van de vorm G.V.L (met G = gebouw, V = verdieping en L = lokaal) en hebben ook een maximaal aantal plaats.\\ 
            \end{tabular}\\
            \caption{FR3.17 -Lokaal detailformulier toevoegen}
            \label{tab:FR3.17 - Lokaal detailformulier toevoegen}
        \end{table}

\textbf{Stappenplan:}

%Lokaal detail formulier verwijderen        
\noindent\begin{table}[H]
            \begin{tabular}{l | p{10cm}}
                \textbf{ID:} & FR3.18 \\ \hline
                \textbf{TITEL:} & 5. Lokaal detail formulier verwijderen\\ \hline
                \textbf{PRIORITEIT:} &  hoog \\ \hline
                \textbf{PREREQUISITIES:} & \\ \hline
                \textbf{TOEGANG:} & Programmabeheerder \\ \hline
                \textbf{BESCHRIJVING:} & De Programmabeheerder moet in staat zijn lokalen te verwijderen (In het geval dat deze niet meer in gebruik worden genomen).\\ 
            \end{tabular}\\
            \caption{FR3.18 - Lokaal detailformulier verwijderen}
            \label{tab:FR3.18 - Lokaal detailformulier verwijderen}
        \end{table}

\textbf{Stappenplan:}
    
\noindent\begin{table}[H]
            \begin{tabular}{l | p{10cm}}
                \textbf{ID:} & FR3.19 \\ \hline
                \textbf{TITEL:} & Lokaaldetail formulier aanpassen\\ \hline
                \textbf{PRIORITEIT:} &  medium \\ \hline
                \textbf{PREREQUISITIES:} & \\ \hline
                \textbf{TOEGANG:} & Programmabeheerder \\ \hline
                \textbf{BESCHRIJVING:} & Mocht er nood zijn aan het aanpassen van een lokaal formulier, moet de Programmabeheerder de optie hebben om dit te doen.\\ 
            \end{tabular}\\
            \caption{FR3.19 - Lokaal detail formulier aanpassen}
            \label{tab:FR3.19 - Lokaaldetail formulier aanpassen}
        \end{table}

\textbf{Stappenplan:}

\noindent\begin{table}[H]
            \begin{tabular}{l | p{10cm}}
                \textbf{ID:} & FR3.20 \\ \hline
                \textbf{TITEL:} & Lesprogramma aanmaken voor een richting\\ \hline
                \textbf{PRIORITEIT:} &  hoog \\ \hline
                \textbf{PREREQUISITIES:} & \\ \hline
                \textbf{TOEGANG:} & Programmabeheerder \\ \hline
                \textbf{BESCHRIJVING:} & De Programmabeheerder is in staat om een lesprogramma manueel aan te maken voor een richting. Deze vraagt dan aan de Scheduler om een lesprogramma op te stellen voor een aantal vakken die bij die richting horen.\\ 
            \end{tabular}\\
            \caption{FR3.20 - Lesprogramma aanmaken voor een richting}
            \label{tab:FR3.20 - Lesprogramma aanmaken voor een richting}
        \end{table}

\textbf{Stappenplan:}

\noindent\begin{table}[H]
            \begin{tabular}{l | p{10cm}}
                \textbf{ID:} & FR3.21 \\ \hline
                \textbf{TITEL:} & Bekijken lesserooster als specifiek gebruiker\\ \hline
                \textbf{PRIORITEIT:} &  laag \\ \hline
                \textbf{PREREQUISITIES:} & \\ \hline
                \textbf{TOEGANG:} & Programmabeheerder \\ \hline
                \textbf{BESCHRIJVING:} & De Programmabeheerder kan via een naam en rolnummer het lesrooster van een specifieke gebruiker bekijken. \\ 
            \end{tabular}\\
            \caption{FR3.21 - Bekijken lesserooster als specifiek gebruiker}
            \label{tab:FR3.21 - Bekijken lesserooster als specifiek gebruiker}
        \end{table}
        
\textbf{Stappenplan:}        
        
\noindent\begin{table}[H]
            \begin{tabular}{l | p{10cm}}
                \textbf{ID:} & FR3.22 \\ \hline
                \textbf{TITEL:} & Bevriezen van een voorstel\\ \hline
                \textbf{PRIORITEIT:} &  medium \\ \hline
                \textbf{PREREQUISITIES:} & \\ \hline
                \textbf{TOEGANG:} & Programmabeheerder \\ \hline
                \textbf{BESCHRIJVING:} & Een voorstel van een professor kan worden worden bevroren en de als het schema word aangemaakt ligt dit voorstel al vast\\ 
            \end{tabular}\\
            \caption{FR3.22 - Bevriezen van een voorstel}
            \label{tab:FR3.22 - Bevriezen van een voorstel}
        \end{table}
        
\textbf{Stappenplan:}
        
\noindent\begin{table}[H]
            \begin{tabular}{l | p{10cm}}
                \textbf{ID:} & FR3.23 \\ \hline
                \textbf{TITEL:} & Bevriezen volledige planning\\ \hline
                \textbf{PRIORITEIT:} &  medium \\ \hline
                \textbf{PREREQUISITIES:} & \\ \hline
                \textbf{TOEGANG:} & Programmabeheerder \\ \hline
                \textbf{BESCHRIJVING:} & Op het moment dat de planning vast is kunnen studenten deze bekijken. Het moet echter nog mogelijk zijn om deze aan te passen en daarom is heeft de Programmabeheerder de optie om de volledige planning te bevriezen en onzichtbaar te maken voor studenten.\\ 
            \end{tabular}\\
            \caption{FR3.23 - Bevriezen volledige planning}
            \label{tab:FR3.23 - Bevriezen volledige planning}
        \end{table}   

\textbf{Stappenplan:}

%Beschikbaarheidsformulier aanmaken   
\noindent\begin{table}[H]
            \begin{tabular}{l | p{10cm}}
                \textbf{ID:} & FR3.24 \\ \hline
                \textbf{TITEL:} & Beschikbaarheidsformulier aanmaken\\ \hline
                \textbf{PRIORITEIT:} &  laag \\ \hline
                \textbf{PREREQUISITIES:} & \\ \hline
                \textbf{TOEGANG:} & Assistent, Professor \\ \hline
                \textbf{BESCHRIJVING:} & Een gebruiker (van het type beschreven in toegang) heeft de optie om een formulier in te dienen waarin hij aangeeft wat zijn vrije momenten zijn om les te geven. De scheduler houdt hier dan rekening mee bij het opstellen van de lesroosters.\\ 
            \end{tabular}\\
            \caption{FR3.24 - Beschikbaarheidsformulier aanmaken}
            \label{tab:FR24 - Beschikbaarheidsformulier aanmaken}
        \end{table}


\textbf{Stappenplan:}