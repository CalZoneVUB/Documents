\subsection{Programmabeheerder Tools - FR6}

%Bekijken lesserooster als specifiek gebruiker
\subsubsection{Bekijken lesserooster als specifiek gebruiker}
\noindent\begin{table}[H]
            \begin{tabular}{l | p{10cm}}
                \textbf{ID:} & FR6.1 \\ \hline
                \textbf{TITEL:} & Bekijken lesserooster als specifiek gebruiker\\ \hline
                \textbf{PRIORITEIT:} &  Laag \\ \hline
                \textbf{PREREQUISITIES:} & \\ \hline
                \textbf{TOEGANG:} & Programmabeheerder \\ \hline
                \textbf{BESCHRIJVING:} & De Programmabeheerder kan via een naam en rolnummer het lesrooster van een specifieke gebruiker bekijken. \\ 
            \end{tabular}\\
            \caption{FR6.1 - Bekijken lesserooster als specifiek gebruiker}
            \label{tab:FR6.1 - Bekijken lesserooster als specifiek gebruiker}
        \end{table}
        
\textbf{Stappenplan:}
	\begin{enumerate}
	\item De gebruiker geeft een rolnummer en naam in
	\item De gebruiker selecteer als welk type hij de interface wil bekijken
	\item De gebruiker krijgt nu het lesrooster van de persson te zien
	\item Er word een extra link onderaan de pagina gezet waarbij de gebruiker zijn interface kan resetten naar deze van de programmabeheerder.
	\end{enumerate}
        
%Bevriezen van een voorstel
\subsubsection{Bevriezen van een voorstel}
\noindent\begin{table}[H]
            \begin{tabular}{l | p{10cm}}
                \textbf{ID:} & FR6.2 \\ \hline
                \textbf{TITEL:} & Bevriezen van een voorstel\\ \hline
                \textbf{PRIORITEIT:} &  Medium \\ \hline
                \textbf{PREREQUISITIES:} & \\ \hline
                \textbf{TOEGANG:} & Programmabeheerder \\ \hline
                \textbf{BESCHRIJVING:} & Een voorstel van een professor kan worden worden bevroren en de als het schema word aangemaakt ligt dit voorstel al vast\\ 
            \end{tabular}\\
            \caption{FR6.2 - Bevriezen van een voorstel}
            \label{tab:FR6.2 - Bevriezen van een voorstel}
        \end{table}
        
\textbf{Stappenplan:}
	\begin{enumerate}
	\item De gebruiker gaat naar zijn gebruikersprofiel via de knop recht op het scherm.
	\item De gebruiker klikt vervolgens op 'Voorstellen' en krijgt hierbij een overzicht van de vakken waaraan hij gekoppeld is. Bij de programmabeheerder zijn dit alle vakken in het systeem.
	\item De gebruiker heeft dan het recht om voorstellen te bekijken en goed of af te keuren.
	\end{enumerate}	        
        
%Bevriezen volledige planning        
\subsubsection{Bevriezen volledige planning}
\noindent\begin{table}[H]
            \begin{tabular}{l | p{10cm}}
                \textbf{ID:} & FR6.3 \\ \hline
                \textbf{TITEL:} & Bevriezen volledige planning\\ \hline
                \textbf{PRIORITEIT:} &  Medium \\ \hline
                \textbf{PREREQUISITIES:} & \\ \hline
                \textbf{TOEGANG:} & Programmabeheerder \\ \hline
                \textbf{BESCHRIJVING:} & Op het moment dat de planning vast is kunnen studenten deze bekijken. Het moet echter nog mogelijk zijn om deze aan te passen en daarom is heeft de Programmabeheerder de optie om de volledige planning te bevriezen en onzichtbaar te maken voor studenten.\\ 
            \end{tabular}\\
            \caption{FR6.3 - Bevriezen volledige planning}
            \label{tab:FR6.3 - Bevriezen volledige planning}
        \end{table}   

\textbf{Stappenplan:}
	\begin{enumerate}
	\item De gebruiker gaat naar zijn gebruikersprofiel via de knop recht op het scherm.
	\item De gebruiker klikt vervolgens op 'Planning' en krijgt hierbij een overzicht van de planning van de geselecteerde richting.
	\item De gebruiker kan dan beslissen om de planning te bevriezen voor heel het rooster.
	\end{enumerate}	

%Toegang gebruiker aanpassen        
\subsubsection{Toegang gebruiker aanpassen }
\noindent\begin{table}[H]
            \begin{tabular}{l | p{10cm}}
                \textbf{ID:} & FR6.4 \\ \hline
                \textbf{TITEL:} & Toegang gebruiker aanpassen \\ \hline
                \textbf{PRIORITEIT:} &  Medium \\ \hline
                \textbf{PREREQUISITIES:} & \\ \hline
                \textbf{TOEGANG:} & Programmabeheerder \\ \hline
                \textbf{BESCHRIJVING:} & Een programmabeheerder kan het soort toegang dat een gebruiker heeft aanpassen, deze bepaalt van welke functionaliteiten een gebruiker allemaal gebruik van mag maken. Er zijn 4 soorten toegangkelijkheden: 
                \begin{itemize}
                		\item Student
                		\item Assistent
                		\item Professor
                		\item Programmabeheerder
                \end{itemize}
            \end{tabular}\\
            \caption{FR6.4 - Toegang gebruiker aanpassen }
            \label{tab:FR6.4 - Toegang gebruiker aanpassen }
        \end{table}   

\textbf{Stappenplan:}
	\begin{enumerate}
	\item De programmabeheerder klikt op de optie "Toegang Aanpassen" en komt op een pagina terecht waarop deze gebruikers kan opzoeken.
	\item De programmabeheerder zoekt een gebruiker op en kiest het type toegang dat deze verkrijgt.
	\item Als laatste klikt hij op \"Aanpassing Toepassen\".
	\end{enumerate}	
	
%TODO 
\textbf{Uitzonderingen:}
\begin{itemize}
\item Een programmabeheerder kan zijn eigen toegang niet aanpassen.
\end{itemize}

\clearpage