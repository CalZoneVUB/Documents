\subsection{Scheduler - FR6}

\noindent\begin{table}[h]
            \begin{tabular}{l | p{10cm}}
                \textbf{ID:} & FR6.1 \\ \hline
                \textbf{TITEL:} & Uitvoeren van een full scheduling\\ \hline
                \textbf{PRIORITEIT:} &  Hoog \\ \hline
                \textbf{PREREQUISITIES:} & \\ \hline
                \textbf{BESCHRIJVING:} & De Scheduler moet in staat zijn om lesroosters op te kunnen stellen, hierbij word er rekening gehouden met verschillende constraint: locatie, tijstip, studenten, professoren en assitenten. \\
            \end{tabular}\\
            \caption{FR6.1}
            \label{tab:mysixteenthtable}
        \end{table}

\noindent\begin{table}[h]
            \begin{tabular}{l | p{10cm}}
                \textbf{ID:} & FR6.2 \\ \hline
                \textbf{TITEL:} & Professor, assistent en programmabeheerder moeten semi-scheduling laten uitvoeren (Haalbaarheid)\\ \hline
                \textbf{PRIORITEIT:} &  medium \\ \hline
                \textbf{PREREQUISITIES:} & \\ \hline
                \textbf{BESCHRIJVING:} & Professoren, assistenten en programmabeheerders kunnen bekijken hoe haalbaar hun schema is. Dit zal een score zijn die afhankelijk is van het aantal beschikbare uren van de persoon en het aantal uren dat deze persoon les zal moeten geven. (Vb: 12uur les te geven en 18u vrij zal een goede haalbaarheidscore krijgen. 16u les en 10u vrij zal een score geven die overeenkomt met onmogelijk)\\
            \end{tabular}\\
            \caption{FR6.2}
            \label{tab:mysixteenthtable}
        \end{table}
        
\noindent\begin{table}[h]
            \begin{tabular}{l | p{10cm}}
                \textbf{ID:} & FR6.3 \\ \hline
                \textbf{TITEL:} & Schedule bevat data, tijdstippen en lokalen voor bepaalde vakken\\ \hline
                \textbf{PRIORITEIT:} &  hoog \\ \hline
                \textbf{PREREQUISITIES:} & \\ \hline
                \textbf{BESCHRIJVING:} & Het schema dat is opgesteld bevat alle informatie over elk vak dat nodig is om het vak te kunnen volgen.\\
            \end{tabular}\\
            \caption{FR6.3}
            \label{tab:mysixteenthtable}
        \end{table}
        
\noindent\begin{table}[h]
            \begin{tabular}{l | p{10cm}}
                \textbf{ID:} & FR6.4 \\ \hline
                \textbf{TITEL:} & Aanpassen berekend rooster \\ \hline
                \textbf{PRIORITEIT:} &  hoog \\ \hline
                \textbf{PREREQUISITIES:} & \\ \hline
                \textbf{BESCHRIJVING:} & Het moet mogelijk zijn om een berekend lessenrooster aan te passen. Aanpassingen kunnen aangebracht worden door de Programmabeheerder.\\
            \end{tabular}\\
            \caption{FR6.4}
            \label{tab:mysixteenthtable}
        \end{table}
        
\noindent\begin{table}[h]
            \begin{tabular}{l | p{10cm}}
                \textbf{ID:} & FR6.5 \\ \hline
                \textbf{TITEL:} & Conflictdetectie\\ \hline
                \textbf{PRIORITEIT:} & hoog \\ \hline
                \textbf{PREREQUISITIES:} & \\ \hline
                \textbf{BESCHRIJVING:} & De Scheduler moet in staat zijn om conflicten te detecteren. Conflicten zijn: 
                \begin{enumerate}
                \item Twee of meer vakken op hetzelfde moment of hetzelfde lokaal
                \item Een professor/assistent  die twee of meer vakken op hetzelfde tijdstip moet geven 
                \item Studenten die twee of meer vakken op hetzelfde tijdstip moeten volgen
                \end{enumerate}
                Puntje 1 en 2 zijn harde conflicten die altijd opgelost moeten worden. Ook verplichte vakken voor studenten van een bepaalde richting mogen nooit samenvallen. Bij keuzevakken is dit wel toegelaten maar niet gewenst.\\
            \end{tabular}\\
            \caption{FR6.5}
            \label{tab:mysixteenthtable}
        \end{table}
        