\subsection{Beveiliging - NFR1}

%Implementatie van HTTPS
\subsubsection{Implementatie van HTTPS}
	\begin{table}[H]
	\caption{NFR1.1 - Implementatie van HTTPS}
    		\begin{tabular}{l | p{10cm}}
        \textbf{ID:} & NFR1.1 \\ \hline
        \textbf{TITEL:} & Implementatie van HTTPS \\ \hline
        \textbf{PRIORITEIT:} &  Medium \\ \hline
        \textbf{PREREQUISITIES:} & Geen\\ \hline
        \textbf{BESCHRIJVING:} & HTTPS gebruiken voor de beveiliging van de applicatie.\\
    \end{tabular} 
	\label{tab:NFR1.1 -Implementatie van HTTPS}
\end{table}

%Hashing
\subsubsection{Hashing}
	\begin{table}[H]
	\caption{NFR1.2 - Hasing}
    		\begin{tabular}{l | p{10cm}}
        \textbf{ID:} & NFR1.2 \\ \hline
        \textbf{TITEL:} & Hashing \\ \hline
        \textbf{PRIORITEIT:} &  Hoog \\ \hline
        \textbf{PREREQUISITIES:} & Geen\\ \hline
        \textbf{BESCHRIJVING:} & Wachtwoorden moeten in de database opgeslagen worden met behulp van een hash. Er zal gebruik gemaakt worden van SHA-256 als cryptografische hash functie en een SALT.\\
    \end{tabular} 
	\label{tab:NFR1.2 -Hashing}
\end{table}

%Session Management
\subsubsection{Session Management}
	\begin{table}[H]
	\caption{NFR1.3 - Session Management}
    		\begin{tabular}{l | p{10cm}}
        \textbf{ID:} & NFR1.3 \\ \hline
        \textbf{TITEL:} & Session Management \\ \hline
        \textbf{PRIORITEIT:} &  Medium \\ \hline
        \textbf{PREREQUISITIES:} & Geen\\ \hline
        \textbf{BESCHRIJVING:} & Elke user-agent krijgt zijn eigen sessie zodat bij het uitloggen alle sessie tupels die overeenkomen met de user ID verwijderd kunnen worden.\\
    \end{tabular} 
	\label{tab:NFR1.3 -Session Management}
\end{table}